\psection {Базовый синтаксис R}
    В данной главе продемонстрирован основной синтаксис и использование R в интерфейсе 
    командной строки. Обычно, пользовательская сессия начинается с ввода \texttt{R} в 
    терминале, либо с запуска приложения с графическим интерфейсом. После этого можно вводить
    команды как в обычной командной строке.\\

    \indent R -- объектно-ориентированный язык программирования. Это значит, что почти всё что 
    угодно можно хранить в R как объект. Каждый объект имеет свой класс. Класс описывает
    содержимое объекта и то, что каждая из функций может с ним делать. Например, функция
    \texttt{plot(x)} производит различные графики, в зависимости от того, является ли 
    \texttt{x}, например, вектором или регрессионной моделью. \\
    
    \indent Оператор присвоения в R выглядит как \texttt{<-}, хотя классический оператор \texttt{=}
    также может быть использован. Две следующие записи эквивалентны: \\
    \indent \texttt{> x <- 2} \\
    \indent \texttt{> x = 2} \\ 
    
    \indent Аргументы в функции передаются внутри круглых скобок. Таким способом можно использовать 
    результат выполнения функции как аргумент другой функции: \\
    \indent \texttt{mean(rnorm(10)\^{}2)} \\ 
    
    \indent C символа \texttt{\#} начинается комментарий до конца строки: \\
    \indent \texttt{\# Это комментарий} \\
    \indent \texttt{2+2 \# Это тоже комментарий} \\ 
    
    \indent R чувствителен к регистру. \texttt{X} и \texttt{x} -- это разные объекты. \\
    
    \indent Рассмотрим некоторые операции присвоения и применения стандартных функций. Как мы
    знаем, объекты могут хранить скалярные и строковые данные: \\
    \indent \texttt{> x <- 2} \\
    \indent \texttt{> x} \\
    \indent \texttt{[1] 2} \\
    \indent \texttt{> x <- "foo"} \\
    \indent \texttt{> x } \\
    \indent \texttt{[1] "foo"} \\ 
    
    \indent Аналогично, объекты могут хранить векторы: \\
    \indent \texttt{> foo <- c(1, 2, 3, 4, 5)} \\
    \indent \texttt{> foo} \\
    \indent \texttt{[1] 1, 2, 3, 4, 5} \\
    \indent \texttt{> bar <- c(6, 7, 8, 9, 10)} \\
    \indent \texttt{> c(foo, bar) \# функция c() - конкатенация объектов} \\
    \indent \texttt{[1] 1, 2, 3, 4, 5, 6, 7, 8, 9, 10} \\ 

    \indent Из этой главы можно сделать вывод, что R так же прост в обращении, как, например, Python. 
    Математические и статистические функции языка R, необходимые для выполнения лабораторных 
    работ, будут описаны в соответствующих главах.
    
