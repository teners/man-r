\psection {Базовый синтаксис R}
    Перед тем, как начать работу в интерпретаторе, давайте ознакомимся с основным 
    синтаксисом и некоторыми полезными функциями языка R. Не стоит пугаться, если в 
    примерах будут использованы незнакомые функции, большинство из них будут описаны
    далее, хотя всегда можно обратиться к документации.

    R -- мультипарадигменный язык, но прежде всего, он объектно-ориентированный. Это 
    значит, что почти всё что угодно можно хранить в R как объект. Каждый объект имеет 
    свой класс. Класс описывает содержимое объекта и то, что каждая из функций может с 
    ним делать. Например, функция \texttt{plot(x)} производит различные графики, в 
    зависимости от того, является ли \texttt{x}, например, вектором или регрессионной моделью. 
    
    Оператор присвоения в R выглядит как \texttt{<-}, хотя классический оператор 
    \texttt{=} также может быть использован. Две следующие записи эквивалентны: \\
    \indent \texttt{> x <- 2} \\
    \indent \texttt{> x = 2} 
    
    Аргументы в функции передаются внутри круглых скобок. Таким способом можно 
    использовать результат выполнения функции как аргумент другой функции: \\
    \indent \texttt{mean(rnorm(10)\^{}2)} 
    
    C символа \texttt{\#} начинается комментарий до конца строки: \\
    \indent \texttt{\# Это комментарий} \\
    \indent \texttt{2+2 \# Это тоже комментарий} 
    
    R чувствителен к регистру. \texttt{X} и \texttt{x} -- это разные объекты. 
    
    Рассмотрим некоторые операции присвоения и применения стандартных функций. Как мы
    знаем, объекты могут хранить скалярные и строковые данные: \\
    \indent \texttt{> x <- 2} \\
    \indent \texttt{> x} \\
    \indent \texttt{[1] 2} \\
    \indent \texttt{> x <- "foo"} \\
    \indent \texttt{> x } \\
    \indent \texttt{[1] "foo"} 
    
    Аналогично, объекты могут хранить векторы: \\
    \indent \texttt{> foo <- c(1, 2, 3, 4, 5)} \\
    \indent \texttt{> foo} \\
    \indent \texttt{[1] 1, 2, 3, 4, 5} \\
    \indent \texttt{> bar <- c(6, 7, 8, 9, 10)} \\
    \indent \texttt{> c(foo, bar) \# функция c() - конкатенация объектов} \\
    \indent \texttt{[1] 1, 2, 3, 4, 5, 6, 7, 8, 9, 10} 

    Так же, в R существуют следующие ключеные слова:
    \begin{itemize}
        \item[--] \texttt{NA} -- <<Not Available>> -- логическая константа, обозначающая
            <<отсутствие>> значения у переменной (объекта);
        \item[--] \texttt{NaN} -- <<Not a Number>> -- например, результат деления нуля на 
            ноль;
        \item[--] \texttt{Inf} и \texttt{-Inf} -- плюс бесконечность и минус бесконечность,
            соответственно.
    \end{itemize}

    Рассмотрим следующий пример:\\
    \indent\texttt{> 0/0} \\
    \indent\texttt{[1] NaN} \\
    \indent\texttt{> 1/0} \\
    \indent\texttt{[1] Inf}

    Условные выражения в R реализованы в привычном if-else синтаксисе. Языковая конструкция
    имеет следующую форму:  
    $$ \texttt{if (\textit{condition}) \textit{expression1} else \textit{expression2}}, $$
    где \textit{condition} -- логическое выражение или значение, \textit{expression1} и 
    \textit{expression2} -- команды, которые выполнятся при истинном и ложном значениях 
    условия соответственно. Стоит отметить, что \texttt{else} -- необязательный оператор.

    Циклы тоже выглядят довольно привычно для любого программиста. Для начала, рассмотрим
    обычный for-цикл: 
    $$ \texttt{for (\textit{name} in \textit{object}) \textit{expression}}, $$
    где \textit{name} -- это переменная цикла, \textit{object} -- объект (например, вектор
    или срезка -- \texttt{object[1:20]}, элементы с 1 по 20), в котором перебираются 
    значения, \textit{expression} -- команда, которая будет выполняться в цикле. \\
    Так же, в R есть цикл while:
    $$ \texttt{while (\textit{condition}) \textit{expression}} $$
    \indent А прежде чем рассматривать оператор \texttt{repeat}, нужно упомянуть ещё пару 
    полезных операторов: \texttt{break} -- оператор прерывания цикла и \texttt{next} -- 
    оператор перехода к следующей итерации цикла. \texttt{repeat} -- немного необычная 
    конструкция. Дело в том, что использование \texttt{break} -- это единственный способ
    завершения repeat-цикла. Чтобы понять принцип его работы, стоит взглянуть на пример
    далее по тексту.

    В дополнение к этим операторам, стоит сказать, что если от условного оператора или цикла
    требуется выполнение нескольких команд, их можно обернуть фигурными скобками, таким 
    образом обозначив отдельный блок. Например: \\
    \indent\texttt{> repeat \{} \\
    \indent\texttt{+ g <- rnorm(1)} \\
    \indent\texttt{+ if (g > 1.0) break} \\ 
    \indent\texttt{+ print(g)} \\
    \indent\texttt{+ \}} \\
    \indent\texttt{[1] -1.214395 0.6393124 0.05505484 -1.217408} 

    Далее, мы рассмотрим некоторые базовые функции, которые пригодятся при выполнении 
    лабораторных работ. \textbf{Примечание}: символ <<...>> в сигнатуре функции означает
    произвольный набор элементов (объектов).
    \begin{mdframed}[style=BadassFrame]
        \texttt{max(..., na.rm = FALSE) \\ min(..., na.rm = FALSE)} \\
        -- вычисляют минимальный и максимальный элементы набора элементов соответственно
        \begin{description}
            \item[na.rm] -- (необязательный) убирать ли из сравнения NA элементы (нет по 
                умолчанию)
        \end{description}
    \end{mdframed}

    \begin{mdframed}[style=BadassFrame]
        \texttt{length(x)} \\
        -- вычисляет длину вектора \texttt{x}
        \begin{description}
           \item[x] -- исходный вектор
        \end{description}
    \end{mdframed}
 
    \begin{mdframed}[style=BadassFrame]
        \texttt{sum(..., na.rm = FALSE)} \\
        -- вычисляет сумму произвольного количества объектов 
    \end{mdframed}

    \begin{mdframed}[style=BadassFrame]
        \texttt{prod(..., na.rm = FALSE)} \\
        -- вычисляет произведение произвольного количества объектов 
    \end{mdframed}

    \begin{mdframed}[style=BadassFrame]
        \texttt{c(..., recursive = FALSE)} \\
        -- осуществляет конкатенацию объектов
        \begin{description}
            \item[recursive] -- (необязательный) разворачивать ли внутренние 
                вектора рекурсивно. Не разворачивает по умолчанию (FALSE)
        \end{description}
    \end{mdframed}

    \indent Из этой главы можно сделать вывод, что R так же прост в обращении, как, например, 
    Python. Математические и статистические функции языка R, необходимые для выполнения 
    лабораторных работ, будут описаны в соответствующих главах.
