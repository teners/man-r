\psection{Лабораторная работа №3}
\textbf{Парная линейная регрессия.}

Парную линейную регрессию строят при изучении связи между исследуемым показателем и
объясняющей переменной. При наличии статистически значимой линейной связи парную
линейную регрессию можно применять для прогнозирования показателя и для оценки влияния
возможных изменений фактора на показатель.

Исходные данные (из полученного .xls файла с вариантом) представляют собой двумерную
выборку $(x_i, y_i, i = 1..30)$. По выборке необходимо построить парную линейную
регрессию, проверить её статистическую значимость и построить точечный и интервальный
прогнозы для значения зависимой переменной.

\begin{enumerate}
	\item Для заданных исходных данных постройте поле корелляции -- диаграмму зависимости
        показателя $y$ от фактора $x$. 

    \item Вычислите выборочные характеристики:
    \begin{itemize}
        \item[--] выборочные средние $\overline{x}$ и $\overline{y}$;
        \item[--] выборочные дисперсии $S_x^{2}$ и $S_y^{2}$;
        \item[--] выборочное среднее квадратические отклонения $S_x$ и $S_y$;
        \item[--] выборочный коэффициент корреляции $r_{xy}$.
    \end{itemize}

    \item Вычислите коэффициенты выборочной линейной регрессии.
\end{enumerate}
\psection{Лабораторная работа №3}
