\unsection{Введение}
\hyphenation{Com-man-der}
\hyphenation{Comp-re-hen-sive R Ar-chi-ve Net-work} 
R – язык программирования и набор программного обеспечения для статистической обработки 
данных с открытым исходным кодом в рамках проекта GNU. R широко используется как 
инструмент для для статистики и интеллектуального анализа данных и фактически является стандартом
для статистических программ. Распространяется в виде исходных кодов и откомпилированных
приложений под Unix/Linux, MacOS и Windows. \\
\indent В основном в R используется интерфейс командой строки, хотя доступны некоторые графические
интерфейсы, которые можно свободно найти в сети (например, пакет R Commander, который так
же свободно распространяется под лицензией GNU GPL). \\ 
\indent R поддерживает широкий спектр статистических и численных методов и обладает хорошей расширяемостью 
с помощью пакетов. Пакеты представляют собой библиотеки для работы специфических функций или 
специальных областей применения. Полный список доступных пакетов, отсортированных по областям 
применения можно найти на сервере Comprehensive R Archive Network (https://cran.r-project.org/web/views/). Так же, R имеет огромные 
возможности для создания качественной графики, например, диаграммы, графики, математические формулы 
(в этих целях авторы рекомендуют использовать не менее популярную систему TeX). \\
\indent Являясь, в первую очередь, интерпретируемым языком, R позволяет сразу же получать ответ на введённую
команду. К примеру, если пользователь вводит "2+2" в командную строку и нажимает enter, сразу же
отображается ответ, что демонстрируется ниже: \\ \\
\indent \texttt{> 2+2} \\ 
\indent \texttt{[1] 4} \\ \\
\indent Как и большинство подобных языков, таких как APL и MATLAB, R поддерживает матричную арифметику.
Одними из примитивных структур данных в R являются списки, векторы, массивы, матрицы, таблцы.
Примерами сложных структур являются регрессионные модели, временные ряды, пространственные 
(географические) координаты и т. д. Стоит отметить, что в R отсутствуют простые типы данных, как,
например, в C или Java. Вместо этого, любая скалярная величина представляется, как вектор длины один.

