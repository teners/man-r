\psection{Лабораторная работа №2}

\textbf{\large Корреляция. Статистическое оценивание.}

\textbf{Корреляция.}

\begin{enumerate}

    \item Смоделируйте выборку из 100 значений нормально распределенной случайной величины с помощью
          функции \texttt{rnorm} (Например, математическое ожидание может быть равным 10, а среднее
          квадратическое отклонение - 2).
    \item Вычислите выборочный коэффициент корреляции для полученных выборок. Для вычисления 
          воспользуйтесь функцией \texttt{cor}. 

          \begin{mdframed}[style=BadassFrame]

              \texttt{cor(var1, var2)} -- вычисляет коэффициент корреляции двух выборок. 
              \begin{description}

                \item[var1] -- первая выборка
                \item[var2] -- вторая выборка
              \end{description}
          \end{mdframed}

    \item Вычислите выборочные коэффициенты корреляции для полученных выборок, используя функцию 
          \texttt{cov}. 

          \begin{mdframed}[style=BadassFrame]

              \texttt{cov(var1, var2)} -- вычисляет коэффициент ковариации двух выборок. 
              \begin{description}

                \item[var1] -- первая выборка
                \item[var2] -- вторая выборка
              \end{description}
          \end{mdframed}

    \item Вычислите средние квадратические отклонения выборок с помощью функции \texttt{sd} и сравните
          с исходными значениями.
\end{enumerate}

\textbf{Точечная и интервальная оценки параметров распределения. Доверительные интервалы.}

