\psection{Лабораторная работа №2}

\textbf{\large Корреляция. Статистическое оценивание.}

\textbf{Корреляция.}

\begin{enumerate}

    \item Смоделируйте 2 выборки из 100 значений нормально распределенной случайной величины. \\
          Для этого получите 100 случайных чисел от 0 до 1 с помощью функции \texttt{runif}.
          Затем, при помощи функции \texttt{qnorm}, которая первым аргументом принимает 
          набор [предварительно сгенерированных] вероятностей, необходимо получить две искомых выборки
          по одному и тому же набору вероятностей с различными значениями матожидания и среднего 
          квадратического отклонения.

          \begin{mdframed}[style=BadassFrame]

              \texttt{runif(num, low, high)} -- генерирует \texttt{num} случайных чисел в заданном интервале $[low; high]$  
              \begin{description}

                \item[num] -- количество чисел
                \item[low] -- нижняя граница интервала
                \item[high] -- верхняя граница интервала
              \end{description}
          \end{mdframed}

          \begin{mdframed}[style=BadassFrame]

              \texttt{qnorm(p, mean = 0, sd = 1, lower.tail = TRUE, log.p = FALSE)} 
                -- вычисляет обратное значение функции нормального распределения по набору вероятностей
              \begin{description}

                \item[p] -- набор вероятностей
                \item[mean] -- (необязательный) матожидание случайной величины
                \item[sd] -- (необязательный) среднее квадратическое отклонение
                \item[lower.tail] -- (необязательный) если TRUE (по умолчанию), вероятности берутся в интервале $P[X \leq x]$. Иначе, $P[X > x]$ 
                \item[log.p] -- (необязательный) если TRUE, предполагается, что вероятности p даны как log(p)
              \end{description}
          \end{mdframed}

    \item Вычислите выборочный коэффициент корреляции для полученных выборок. Для вычисления 
          воспользуйтесь функцией \texttt{cor}. Достаточно использовать первые два аргумента.

          \begin{mdframed}[style=BadassFrame]

              \texttt{cor(x, y = NULL, use = ``everything'', \\ method = c(``pearson'', ``kendall'', ``spearman'')} -- вычисляет коэффициент корреляции двух наборов величин. 
              \begin{description}

                \item[x] -- первый набор
                \item[y] -- (необязательный) второй набор (аргумент по умолчанию эквивалентен y=x, но более эффективен )
                \item[use] -- (необязательный) метод вычисления коэффициента ковариации в случае отсутствия некоторых значений.
                    ``everything'' (по умолчанию), ``all.obs", ``complete.obs'', ``na.or.complete'', or ``pairwise.complete.obs''
                \item[method] -- (необязательный) способ вычисления коэффициента ковариации. Должен быть один из: 
                    ``pearson'' (по умолчанию), ``kendall'', ``spearman''
              \end{description}
          \end{mdframed}

    \item Вычислите выборочные коэффициенты корреляции для полученных выборок, используя функцию 
          \texttt{cov}. Достаточно использовать первые два аргумента.

          \begin{mdframed}[style=BadassFrame]

              \texttt{cov(x, y = NULL, use = ``everything'', \\ method = c(``pearson'', ``kendall'', ``spearman''))} 
                    -- вычисляет коэффициент ковариации двух наборов величин. 
              \begin{description}

                \item[x] -- первый набор
                \item[y] -- (необязательный) второй набор (аргумент по умолчанию эквивалентен y=x, но более эффективен )
                \item[use] -- (необязательный) метод вычисления коэффициента ковариации в случае отсутствия некоторых значений.
                    ``everything'' (по умолчанию), ``all.obs'', ``complete.obs'', ``na.or.complete'', or ``pairwise.complete.obs''
                \item[method] -- (необязательный) способ вычисления коэффициента ковариации. Должен быть один из: 
                    ``pearson'' (по умолчанию), ``kendall'', ``spearman''
              \end{description}
          \end{mdframed}

    \item Вычислите средние квадратические отклонения выборок с помощью функции \texttt{sd} и сравните
          с исходными значениями.


    \textbf{Точечная и интервальная оценки параметров распределения. Доверительные интервалы.}
    \item Смоделируйте выборку из 100 значений нормально распределенной случайной величины с помощью
          функции \texttt{rnorm}. Сохраните значения выбранных математического ожидания и 
          среднего квадратического отклонения. Также вычислите дисперсию на основании математического 
          ожидания.

    \item Вычислите дисперсию и среднее квадратическое отклонение выборки с помощью 
          функций \texttt{var} и \texttt{sd}. Сравните полученные значения с ранее использованными.

          \begin{mdframed}[style=BadassFrame]

              \texttt{var(x, y = NULL, na.rm = FALSE, use)} -- вычисляет дисперсию набора величин
              \begin{description}

                \item[x] -- первый набор
                \item[y] -- (необязательный) второй набор (аргумент по умолчанию эквивалентен y=x, но более эффективен )
                \item[na.rm] -- (необязательный) должны ли учитываться отсутствующие значения? (нет по умолчанию).
                \item[use] -- (необязательный) метод вычисления коэффициента ковариации в случае отсутствия некоторых значений.
                    ``everything'' (по умолчанию), ``all.obs", ``complete.obs'', ``na.or.complete'', or ``pairwise.complete.obs''
              \end{description}
          \end{mdframed}

    \item Вычислите вероятность попадания случайной величины в интервал $[5;11]$. Для нахождения значений функции
          нормальной случайной величины воспользуйтесь функцией \texttt{pnorm} соответствующим образом: $P\{5<X<11\}=F(11)-F(5)$.

          \begin{mdframed}[style=BadassFrame]

              \texttt{pnorm(q, mean = 0, sd = 1, lower.tail = TRUE, log.p = FALSE)} 
                -- вычисляет результат функции нормального распределения
              \begin{description}

                \item[q] -- аргумент функции нормального распределения
                \item[mean] -- (необязательный) матожидание случайной величины
                \item[sd] -- (необязательный) среднее квадратическое отклонение
                \item[lower.tail] -- (необязательный) если TRUE (по умолчанию), вероятности берутся в интервале $P[X \leq x]$. Иначе, $P[X > x]$ 
                \item[log.p] -- (необязательный) если TRUE, предполагается, что вероятности p даны как log(p)
              \end{description}
          \end{mdframed}

    \item Получите случайную выборку на 10 элементов из вышеуказанной генеральной совокупности при помощи функции
          \texttt{sample}

          \begin{mdframed}[style=BadassFrame]

              \texttt{sample(x, size, replace = FALSE, prob = NULL)} 
                -- позволяет получить \texttt{size} случайных значений из набора \texttt{x}
              \begin{description}

                \item[x] -- входной набор значений
                \item[size] -- количество (положительное) случайных значений
                \item[replace] -- (необязательный) слудует ли выбирать с замещением элемента
                \item[prob] -- (необязательный) вектор значений вероятности выборки элементов из входного набора значений
              \end{description}
          \end{mdframed}


\end{enumerate}



