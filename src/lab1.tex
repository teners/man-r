\psection{Лабораторная работа №1}
\textbf{Основы обработки статистических данных в R. \\ \indent Выборочные характеристики.}
\begin{enumerate}
	\item Получите две выборки объемом 50 и 30 значений из нормально распределенных генеральных совокупностей с разными значениями параметров.\\
		Для этого стоит воспользоваться функцией \texttt{rnorm}.\\
		Она принимает три агрумента:
		\begin{itemize}
			\item[--] Количество чисел;
			\item[--] Математическое ожидание, mean, значением по умолчанию является 0; 
			\item[--] Среднеквадратическое отклонение, sd, значением по умолчанию является 1.
		\end{itemize} 
		Пример использования:\\
		\indent \texttt{>rnorm(4, mean=10, sd=3)}\\
		\indent \texttt{[1] 12.603994 10.818511  9.115428  6.367522}

	\item По объединенной выборке, моделирующей выборку из конечной смеси двух распределений, 
		составьте вариационный ряд (выборку, упорядоченную по возрастанию значений).
		Конкатенацию векторов можно осуществить при помощи функции \texttt{c()}.\\
		Затем объединенную выборку можно отсортировать используя функцию \texttt{sort()}. 
		Для этого достаточно передать ей, как аргумент, ваш вектор. Пример:\\
		\indent \texttt{> y} \\
 		\indent \texttt{[1]  3.064658  0.952703  2.458550  7.531425  4.424056 -1.004320  9.993588} \\
 		\indent \texttt{[8]  6.985664  4.892350  1.910246} \\
        \indent \texttt{> y <- sort(y)} \\
		\indent \texttt{> y} \\
		\indent \texttt{ [1] -1.004320  0.952703  1.910246  2.458550  3.064658  4.424056  4.892350} \\
		\indent \texttt{ [8]  6.985664  7.531425  9.993588}

	\item Найдите выборочные характеристики, в скобках указаны функции, которые следует использовать для получения значения:
		\begin{itemize}
			\item[--] Наибольшее значение, \texttt{(min)};
			\item[--] Наименьшее значение, \texttt{(max)};
			\item[--] Объем выборки, \texttt{(length)};
			\item[--] Выборочную среднюю, \texttt{(ave)};
			\item[--] Медиану: значение, которое делит вариационный ряд на две равные (по числу значений) части, \texttt{(median)};
			\item[--] Вариационный размах (размах выборки): разница между наибольшим и наименьшим значениями выборки, \texttt{(max - min)}; 
			\item[--] Исправленную (несмещенную) выборочную дисперсию, \texttt{(var)};
			\item[--] Выборочное среднее квадратическое отклонение, вычисленное по исправленной (несмещенной) выборочной дисперсии, \texttt{(sd)};
		\end{itemize}
		Все перечисленные выше функции достаточно вызвать с одним аргментом - вашим вектором значений.

	\item Постройте гистограмму частот распределения значений показателя.
		Сначала необходимо разбить диапазон изменений показателя на конечное число $k$
		непересекающихся интервалов $J_{i}, i=1,..,k$  и для каждого из них 
		подсчитать частоту попаданий значений показателя $l_{i}, i=\overline{1,n}$
		(т.е. число значений объединенной выборки, попавших в интервал).
		$k$ определяется формулой Стерджеса: $k=[1+\log_2 n]$.
		Для этого стоит использовать функцию \texttt{log2}:
		\begin{mdframed}[style=BadassFrame]
			\texttt{log2(x)}\\
			-- вычисляет логарифм числа x по основанию 2.
		\end{mdframed}
		Для получения правильного значения требуется написать простое ветвление для окргуления значения в большую сторону
		(подразумевается, что y.k - и есть коэффициент, вычисленный по формуле Стерджеса):\\
		\texttt{if (y.k > round(y.k)) y.k <- round(y.k) + 1 else y.k <- round(y.k)}\\
		
		Дальше построения гистограммы нам потребуется вычислить границы интервалов:
		\begin{itemize}
			\item[--] Вычислим шаг для интервала (y.delta - размах выборки):\\
				\texttt{y.step<-y.delta/y.k}
			\item[--] Подготовим вектор шагов:\\
				\texttt{y.steps<-(seq(1:(y.k+1))-1)*y.step}
			\item[--] Иницализиуем объект y.breaks:\\
				\texttt{y.breaks<-seq(0:8)}
			\item[--] Получим точные границы интервалов:\\
				\texttt{for (i in seq(1:(y.k+1))) y.breaks[i] <- y.steps[i] + y[1]}
		\end{itemize}

		Построим искомую гистограмму с помощью функции \texttt{hist}:
		\begin{mdframed}[style=BadassFrame]
			\texttt{hist(x, breaks=br)} \\
			-- строит гистограмму на основе вектора x, с разделителями br.
		\end{mdframed}

		Для удобства стоит сохранить ее, например в \texttt{y.hist1}.
		В дальнейшем можно получить найденные частоты обратившись к вектору \texttt{y.hist1\$counts}.

	\item Вычислите относительные частоты попадания в каждый интервал разбиения.
		Относительная частота  равна значению соответствующей частоты, деленной на число элементов выборки: 
		$p_{i} = \frac{l_{i}}{n}, i=1,..,k$ .
		Сумма значений полученных относительных частот должна быть равна 1.\\
		Вычислим относительные частоты (N - объем вашей выборки):\\
		\texttt{y.rel\_fr<-y.hist1\$counts/N} \\
		Проверим их сумму:\\
		\texttt{sum(y.rel\_fr)}

	\item Подсчитайте кумулятивные (интегральные) частоты.
		Значение кумулятивной частоты для интервала представляет собой сумму частот 
		текущего и всех предыдущих интервалов (накопленные частоты):
		$F_{i}=\sum_{j=1}^i l_{i}, i=1,..,k$ .\\
		\texttt{y.cum\_fr<-seq(1:y.k)}\\
		\texttt{for (i in seq(1:y.k)) y.cum\_fr[i] <- sum(y.hist\$counts[1:i])}


	\item Вычислите относительные кумулятивные частоты,
		которые равны кумулятивным частотам, деленным на число элементов выборки.
		$F_{i}=\frac{\sum_{j=1}^i l_{i}}{n}, i=1,..,k$ .\\
		\texttt{y.cum\_rel\_fr<-seq(1:y.k)} \\
		\texttt{for (i in seq(1:y.k)) y.cum\_rel\_fr[i] <- sum(y.rel\_fr[1:i])}

\end{enumerate}
