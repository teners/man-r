\psection{Лабораторная работа №1}
\textbf{Основы обработки статистических данных в R. \\ \indent Выборочные характеристики.}
\begin{enumerate}
	\item Получите две выборки объемом 50 и 30 значений из нормально распределенных генеральных совокупностей с разными значениями параметров.\\
		Для этого стоит воспользоваться функцией \texttt{rnorm}.\\
		Она принимает три агрумента:
		\begin{itemize}
			\item[--] Количество чисел;
			\item[--] Математическое ожидание, mean, значением по умолчанию является 0; 
			\item[--] Среднеквадратическое отклонение, sd, значением по умолчанию является 1.
		\end{itemize} 
		Пример использования:\\
		\indent \texttt{>rnorm(4, mean=10, sd=3)}\\
		\indent \texttt{[1] 12.603994 10.818511  9.115428  6.367522}

	\item По объединенной выборке, моделирующей выборку из конечной смеси двух распределений, 
		составьте вариационный ряд (выборку, упорядоченную по возрастанию значений).
		Конкатенацию векторов можно осуществить при помощи функции \texttt{c()}.\\
		Затем объединенную выборку можно отсортировать используя функцию \texttt{sort()}. 
		Для этого достаточно передать ей, как аргумент, ваш вектор. Пример:\\
		\indent \texttt{> y} \\
 		\indent \texttt{[1]  3.064658  0.952703  2.458550  7.531425  4.424056 -1.004320  9.993588} \\
 		\indent \texttt{[8]  6.985664  4.892350  1.910246} \\
        \indent \texttt{> y <- sort(y)} \\
		\indent \texttt{> y} \\
		\indent \texttt{ [1] -1.004320  0.952703  1.910246  2.458550  3.064658  4.424056  4.892350} \\
		\indent \texttt{ [8]  6.985664  7.531425  9.993588}

	\item Найдите выборочные характеристики, в скобках указаны функции, которые следует использовать для получения значения:
		\begin{itemize}
			\item[--] Наибольшее значение, \texttt{(min)};
			\item[--] Наименьшее значение, \texttt{(max)};
			\item[--] Объем выборки, \texttt{(length)};
			\item[--] Выборочную среднюю, \texttt{(ave)};
			\item[--] Медиану: значение, которое делит вариационный ряд на две равные (по числу значений) части, \texttt{(median)};
			\item[--] Вариационный размах (размах выборки): разница между наибольшим и наименьшим значениями выборки, \texttt{(max - min)}; 
			\item[--] Исправленную (несмещенную) выборочную дисперсию, \texttt{(var)};
			\item[--] Выборочное среднее квадратическое отклонение, вычисленное по исправленной (несмещенной) выборочной дисперсии, \texttt{(sd)};
		\end{itemize}
		Все перечисленные выше функции достаточно вызвать с одним аргментом - вашим вектором значений.

	\item Постройте гистограмму частот распределения значений показателя.

	\item Вычислите относительные частоты попадания в каждый интервал разбиения.

	\item Подсчитайте кумулятивные (интегральные) частоты.

	\item Вычислите относительные кумулятивные частоты, которые равны кумулятивным частотам, деленным на число элементов выборки.

\end{enumerate}
