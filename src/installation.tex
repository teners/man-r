\psection{Инсталляция R}
    Прежде чем начать работу с R, необходимо установить интерпретатор языка. В данной главе рассматриваются 
    основные способы различных платформах. \textbf{Примечание}: указанные ссылки ведут на 
    российское зеркало \url{https://cran.r-project.org}.

    \begin{enumerate}
        \item \textbf{Семейство дистрибутивов GNU/Linux}
            \begin{itemize}
                \item[--] APT (aptitude):
                    \texttt{sudo apt-get install r-base}
                
                \item[--] RPM:
                    \texttt{sudo yum install R} \\
                    RPM пакеты можно найти по ссылке \url{https://cran.gis-lab.info/bin/linux/}
                
                \item[--] Исходный код R можно так же найти на сайте CRAN (\url{https://cran.gis-lab.info/}).
            \end{itemize}
        
        \item \textbf{Windows} \\
            Для установки интерпретатора R под Windows на сайте CRAN (\url{https://cran.gis-lab.info/bin/windows/base/})
            можно найти ссылку на установочный файл и инструкции по установке.
        
        \item \textbf{Mac OS} \\
            Пользователям Mac OS, как и пользователям Windows, предлагается скачать бинарный пакет и прочитать инструкии
            по установке на сайте CRAN (\url{https://cran.gis-lab.info/bin/macosx/}).
    \end{enumerate} 
