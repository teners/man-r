\psection{Работа в интерпретаторе R}
    Теперь можно приступить непосредственно к работе с R в интерпретаторе. Обычно, 
    пользовательская сессия начинается со ввода команды \texttt{R} (или команды, 
    которая соответствует вашей операционной системе) в терминале, либо с запуска 
    интегрированной среды разработки. После этого команды вводятся как в обычной
    командной строке. К слову, для выхода из интерпретатора необходимо использовать
    функцию \texttt{quit()} или её синоним -- \texttt{q()}. Интерпретатор предложит
    сохранить текущий сеанс, это позволит просмотреть историю команд и сохранить
    текущие значения объектов при запуске интерпретатора в следующий раз.

    Прежде всего, интерпретатор R можно использовать как простой калькулятор: \\
    \indent\texttt{> 2 * 21} \\
    \indent\texttt{[1] 42} \\
    \indent\texttt{> log(8 ** 5, 64 / 2)} \\
    \indent\texttt{[1] 3}

    При выполнении лабораторных работ придётся работать с переменными, векторами,
    таблцами, функциями. Это было описано в предыдущей главе, поэтому эта часть в
    данной главе опускается.

    Отдельно стоит упомянуть ещё несколько полезных при выполнении лабораторных работ 
    функций. Они подойдут для консольной версии интерпретатора, способы работы с 
    графической версией опущены ввиду различий между разнообразными средами 
    разработки (к тому же, они должны быть описаны в официальной документации среды 
    разработки). Сохранить введённые во время сессии в интерпретаторе команды можно 
    функцией \texttt{savehistory(file="\textit{filename}")}, а загрузить этот файл 
    можно функций \texttt{source("\textit{filename}")}. Давайте рассмотрим пример 
    такой сессии: \\
    \indent\texttt{machine@user \$ R} \\
    \indent\texttt{> x <- 10} \\
    \indent\texttt{> y <- 15} \\
    \indent\texttt{> z <- x + y} \\
    \indent\texttt{> z} \\
    \indent\texttt{[1] 25} \\
    \indent\texttt{> savehistory(file="sum.R")} \\
    \indent\texttt{> q()} \\
    \indent\texttt{machine@user \$ R} \\
    \indent\texttt{> source("sum.R")} \\
    \indent\texttt{> z} \\
    \indent\texttt{[1] 25}

    В дополнение, рассмотрим функции для работы с объектами, созданными в текущей
    пользовательской сессии. Начнём с функции \texttt{ls()}. Данная функция 
    позволяет вывести на экран все объекты, инициализированные какими-либо данными.
    Функция \texttt{rm(..., \textit{list})} удаляет перечисленный набор переменных 
    произвольного размера, а так же, имеет необязательный параметр \textit{list},
    в котором может храниться вектор или список имён переменных. Данный функционал
    будет продемонстрирован далее в примере. \\
    \indent\texttt{> x <- 5; y <- 6; z <- 7} \\
    \indent\texttt{> ls()} \\
    \indent\texttt{[1] "x" "y" "z"} \\
    \indent\texttt{> rm(x)} \\
    \indent\texttt{> ls()} \\
    \indent\texttt{[1] "y" "z"} \\
    \indent\texttt{> rm(list = ls())} \\
    \indent\texttt{> ls()} \\
    \indent\texttt{[1] character(0) \# в текущей сессии нет объектов}

    Завершая главу, нельзя не упомянуть функции, позволяющие получить справку по
    функциям языка R. \texttt{help(\textit{function})} открывает документацию по
    функции \textit{function}, а функция \texttt{RSiteSearch("\textit{query}")}
    открывает интернет-браузер и производит запрос \textit{query} на сайте 
    R-project.
