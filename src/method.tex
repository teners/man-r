\documentclass[a4paper,12pt]{article} %размер бумаги устанавливаем А4, шрифт 12пунктов
\usepackage[T2A]{fontenc}
\usepackage[utf8]{inputenc}	%кодировка
\usepackage[english,russian]{babel} %используем русский и английский языки с переносами

\usepackage{amsmath}

\usepackage{geometry} % Меняем поля страницы
\geometry{a4paper, left=2cm, right=1.5cm, top=1cm, bottom=2cm }

\setlength{\parindent}{1cm}
\setlength{\parskip}{1ex plus 0.5ex minus 0.2ex}


\begin{document}

\begin{titlepage}

	\begin{center}
		\sc{МИНИСТЕРСТВО ОБРАЗОВАНИЯ И НАУКИ РОССИЙСКОЙ ФЕДЕРАЦИИ 
		ФЕДЕРАЛЬНОЕ ГОСУДАРСТВЕННОЕ АВТОНОМНОЕ ОБРАЗОВАТЕЛЬНОЕ УЧРЕЖДЕНИЕ
		ВЫСШЕГО ОБРАЗОВАНИЯ
		«САНКТ-ПЕТЕРБУРГСКИЙ ГОСУДАРСТВЕННЫЙ УНИВЕРСИТЕТ
		АЭРОКОСМИЧЕСКОГО ПРИБОРОСТРОЕНИЯ»  }

		\vspace{2cm}
		\sc{Авторы: Д.В. Дулин, С.А. Соколов}\\
		{\footnotesize Под редакцией М.А. Нарбута и М.В. Фаттаховой}
	\end{center} 

	\begin{center}
		\begin{tabular}{p{13cm}}
			\vspace{2cm}\\
			\begin{center}
				\vspace{1cm}
				\huge{Методические указания для выполнения лабораторных работ в среде R по курсу прикладной теории вероятностей.}
			\end{center}
		\end{tabular} 
	\end{center}

	\begin{center}
		\vspace{8cm}
		Санкт-Петербург\\
		2016
	\end{center}

\end{titlepage}


\tableofcontents 
\newpage

\section{Введение}



\end{document}
