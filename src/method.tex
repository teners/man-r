\documentclass[a4paper,12pt]{article} %размер бумаги устанавливаем А4, шрифт 12пунктов
\usepackage[T2A]{fontenc}
\usepackage[utf8]{inputenc}	%кодировка
\usepackage[english,russian]{babel} %используем русский и английский языки с переносами
\usepackage[colorlinks=true,urlcolor=black,linkcolor=black,filecolor=black,citecolor=black]{hyperref}
\usepackage{lipsum}
\usepackage{fancyhdr}
%% set indent at the beginning of a paragraph
\usepackage{indentfirst} 
\usepackage[titletoc]{appendix}
\usepackage{bookmark}
\usepackage{enumitem}
\usepackage{listings}
%% set urls' colors 
\usepackage{amsmath}

\usepackage{geometry} % Меняем поля страницы
\geometry{a4paper, left=2cm, right=1.5cm, top=1cm, bottom=2cm }

\setlength{\parindent}{1cm}
\setlength{\parskip}{1ex plus 0.5ex minus 0.2ex}

%% macro starts new section on new page
\newcommand{\psection}[1]{\newpage\section{#1}}
%% this one does the same but for unenumerate sections
\newcommand{\unsection}[1]{\newpage\section*{#1}\addcontentsline{toc}{section}{#1}}

\begin{document}

\begin{titlepage}

	\begin{center}
		\sc{МИНИСТЕРСТВО ОБРАЗОВАНИЯ И НАУКИ РОССИЙСКОЙ ФЕДЕРАЦИИ 
		ФЕДЕРАЛЬНОЕ ГОСУДАРСТВЕННОЕ АВТОНОМНОЕ ОБРАЗОВАТЕЛЬНОЕ УЧРЕЖДЕНИЕ
		ВЫСШЕГО ОБРАЗОВАНИЯ
		«САНКТ-ПЕТЕРБУРГСКИЙ ГОСУДАРСТВЕННЫЙ УНИВЕРСИТЕТ
		АЭРОКОСМИЧЕСКОГО ПРИБОРОСТРОЕНИЯ»  }

		\vspace{2cm}
		\sc{Авторы: В.В Голубев, Д.В. Дулин, С.А. Соколов}\\
	\end{center} 

	\begin{center}
		\begin{tabular}{p{13cm}}
			\vspace{2cm}\\
			\begin{center}
				\vspace{1cm}
				\huge{Методические указания для выполнения лабораторных работ в среде R по курсу прикладной теории вероятностей.}
			\end{center}
		\end{tabular} 
	\end{center}

	\begin{center}
		\vspace{8cm}
		Санкт-Петербург\\
		2016
	\end{center}

\end{titlepage}


\tableofcontents 
\newpage

\unsection{Введение}
    \hyphenation{Com-man-der}
    \hyphenation{Comp-re-hen-sive R Ar-chi-ve Net-work} 

    R – язык программирования и набор программного обеспечения для статистической обработки 
    данных с открытым исходным кодом в рамках проекта GNU. R широко используется как 
    инструмент для для статистики и интеллектуального анализа данных и фактически является стандартом
    для статистических программ. Распространяется в виде исходных кодов и откомпилированных
    приложений под Unix/Linux, MacOS и Windows. 
    
    \indent В основном в R используется интерфейс командой строки, хотя доступны некоторые графические
    интерфейсы, которые можно свободно найти в сети (например, пакет R Commander, который так
    же свободно распространяется под лицензией GNU GPL).  
    
    \indent R поддерживает широкий спектр статистических и численных методов и обладает хорошей расширяемостью 
    с помощью пакетов. Пакеты представляют собой библиотеки для работы специфических функций или 
    специальных областей применения. Полный список доступных пакетов, отсортированных по областям 
    применения можно найти на сервере Comprehensive R Archive Network (\textbf{https://cran.r-project.org/web/views/}). Так же, R имеет огромные 
    возможности для создания качественной графики, например, диаграммы, графики, математические формулы 
    (в этих целях авторы рекомендуют использовать не менее популярную систему TeX). 
    
    \indent R является интерпретируемым языком, что позволяет сразу же получать ответ на введённую
    команду. К примеру, если пользователь вводит \texttt{2+2} в командную строку и нажимает enter, сразу же
    отображается ответ, что демонстрируется ниже: \\
    \indent \texttt{> 2+2} \\ 
    \indent \texttt{[1] 4} 
    
    \indent Как и большинство подобных языков, таких как APL и GNU Octave, R поддерживает матричную арифметику.
    Одними из примитивных структур данных в R являются списки, векторы, массивы, матрицы, таблцы.
    Примерами сложных структур являются регрессионные модели, временные ряды, пространственные 
    (географические) координаты и т. д. Стоит отметить, что в R отсутствуют простые типы данных, как,
    например, в C или Java. Вместо этого, любая скалярная величина представляется, как вектор длины один.  
    
    \indent Документацию по R на русском языке можно найти здесь: \url{http://herba.msu.ru/shipunov/software/r/r-ru.htm}. \\
    \indent И на английском языке здесь: \url{https://stat.ethz.ch/R-manual/R-devel/doc/html/}.

\psection{Инсталляция R}
В данной главе рассматриваются основные способы установки интерпретатора языка R на
различных платформах. \textbf{Примечание}: указанные ссылки ведут на российское зеркало \url{https://cran.r-project.org}.
\begin{enumerate}
    \item Семейство дистрибутивов GNU/Linux
        \begin{itemize}
            \item[--] APT (aptitude):
                \texttt{sudo apt-get install r-base}
            \item[--] RPM:
                \texttt{sudo yum install R} \\
                RPM пакеты можно найти по ссылке \url{https://cran.gis-lab.info/bin/linux/}
            \item[--] Исходный код R можно так же найти на сайте CRAN -- \url{https://cran.r-project.org/}
        \end{itemize}
    \item Windows
        Для установки интерпретатора R под Windows на сайте CRAN \url{https://cran.gis-lab.info/bin/windows/base/}
        можно найти ссылку на установочный файл и инструкции по установке.
    \item Mac OS
        Пользователям Mac OS, как и пользователям Windows предлагается скачать бинарный пакет и прочитать инструкии
        по установке на сайте CRAN \url{https://cran.gis-lab.info/bin/macosx/}.
\end{enumerate} 

\psection {Базовый синтаксис R}
    В данной главе продемонстрирован основной синтаксис и использование R в интерфейсе 
    командной строки. Обычно, пользовательская сессия начинается с ввода \texttt{R} в 
    терминале, либо с запуска приложения с графическим интерфейсом. После этого можно вводить
    команды как в обычной командной строке.\\

    \indent R -- объектно-ориентированный язык программирования. Это значит, что почти всё что 
    угодно можно хранить в R как объект. Каждый объект имеет свой класс. Класс описывает
    содержимое объекта и то, что каждая из функций может с ним делать. Например, функция
    \texttt{plot(x)} производит различные графики, в зависимости от того, является ли 
    \texttt{x}, например, вектором или регрессионной моделью. \\
    
    \indent Оператор присвоения в R выглядит как \texttt{<-}, хотя классический оператор \texttt{=}
    также может быть использован. Две следующие записи эквивалентны: \\
    \indent \texttt{> x <- 2} \\
    \indent \texttt{> x = 2} \\ \\
    
    \indent Аргументы в функции передаются внутри круглых скобок. Таким способом можно использовать 
    результат выполнения функции как аргумент другой функции: \\
    \indent \texttt{mean(rnorm(10)\^{}2)} \\ \\
    
    \indent C символа \texttt{\#} начинается комментарий до конца строки: \\
    \indent \texttt{\# Это комментарий} \\
    \indent \texttt{2+2 \# Это тоже комментарий} \\ \\
    
    \indent R чувствителен к регистру. \texttt{X} и \texttt{x} -- это разные объекты. \\
    
    \indent Рассмотрим некоторые операции присвоения и применения стандартных функций. Как мы
    знаем, объекты могут хранить скалярные и строковые данные: \\
    \indent \texttt{> x <- 2} \\
    \indent \texttt{> x} \\
    \indent \texttt{[1] 2} \\
    \indent \texttt{> x <- "foo"} \\
    \indent \texttt{> x } \\
    \indent \texttt{[1] "foo"} \\ \\
    
    \indent Аналогично, объекты могут хранить векторы: \\
    \indent \texttt{> foo <- c(1, 2, 3, 4, 5)} \\
    \indent \texttt{> foo} \\
    \indent \texttt{[1] 1, 2, 3, 4, 5} \\
    \indent \texttt{> bar <- c(6, 7, 8, 9, 10)} \\
    \indent \texttt{> c(foo, bar) \# функция c() - конкатенация объектов} \\
    \indent \texttt{[1] 1, 2, 3, 4, 5, 6, 7, 8, 9, 10} \\ \\

    \indent Из этой главы можно сделать вывод, что R так же прост в обращении, как, например, Python. 
    Математические и статистические функции языка R, необходимые для выполнения лабораторных 
    работ, будут описаны в соответствующих главах.
    

\psection{Лабораторная работа №1}
\textbf{Основы обработки статистических данных в R. \\ \indent Выборочные характеристики.}
\begin{enumerate}
	\item Получите две выборки объемом 50 и 30 значений из нормально распределенных генеральных совокупностей с разными значениями параметров.\\
		Для этого стоит воспользоваться функцией \texttt{rnorm}.\\
		Она принимает три агрумента:
		\begin{itemize}
			\item[--] Количество чисел;
			\item[--] Математическое ожидание, mean, значением по умолчанию является 0; 
			\item[--] Среднеквадратическое отклонение, sd, значением по умолчанию является 1.
		\end{itemize} 
		Пример использования:\\
		\indent \texttt{>rnorm(4, mean=10, sd=3)}\\
		\indent \texttt{[1] 12.603994 10.818511  9.115428  6.367522}

	\item По объединенной выборке, моделирующей выборку из конечной смеси двух распределений, 
		составьте вариационный ряд (выборку, упорядоченную по возрастанию значений).
		Конкатенацию векторов можно осуществить при помощи функции \texttt{c()}.\\
		Затем объединенную выборку можно отсортировать используя функцию \texttt{sort()}. 
		Для этого достаточно передать ей, как аргумент, ваш вектор. Пример:\\
		\indent \texttt{> y} \\
 		\indent \texttt{[1]  3.064658  0.952703  2.458550  7.531425  4.424056 -1.004320  9.993588} \\
 		\indent \texttt{[8]  6.985664  4.892350  1.910246} \\
        \indent \texttt{> y <- sort(y)} \\
		\indent \texttt{> y} \\
		\indent \texttt{ [1] -1.004320  0.952703  1.910246  2.458550  3.064658  4.424056  4.892350} \\
		\indent \texttt{ [8]  6.985664  7.531425  9.993588}

	\item Найдите выборочные характеристики, в скобках указаны функции, которые следует использовать для получения значения:
		\begin{itemize}
			\item[--] Наибольшее значение, \texttt{(min)};
			\item[--] Наименьшее значение, \texttt{(max)};
			\item[--] Объем выборки, \texttt{(length)};
			\item[--] Выборочную среднюю, \texttt{(ave)};
			\item[--] Медиану: значение, которое делит вариационный ряд на две равные (по числу значений) части, \texttt{(median)};
			\item[--] Вариационный размах (размах выборки): разница между наибольшим и наименьшим значениями выборки, \texttt{(max - min)}; 
			\item[--] Исправленную (несмещенную) выборочную дисперсию, \texttt{(var)};
			\item[--] Выборочное среднее квадратическое отклонение, вычисленное по исправленной (несмещенной) выборочной дисперсии, \texttt{(sd)};
		\end{itemize}
		Все перечисленные выше функции достаточно вызвать с одним аргментом - вашим вектором значений.

	\item Постройте гистограмму частот распределения значений показателя.
		Сначала необходимо разбить диапазон изменений показателя на конечное число $k$
		непересекающихся интервалов $J_{i}, i=1,..,k$  и для каждого из них 
		подсчитать частоту попаданий значений показателя $l_{i}, i=\overline{1,n}$
		(т.е. число значений объединенной выборки, попавших в интервал).
		$k$ определяется формулой Стерджеса: $k=[1+\log_2 n]$.
		Для этого стоит использовать функцию \texttt{log2}:
		\begin{mdframed}[style=BadassFrame]
			\texttt{log2(x)} -- вычисляет логарифм числа x по основанию 2.
		\end{mdframed}
		Для получения правильного значения требуется написать простое ветвление для окргуления значения в большую сторону
		(подразумевается, что y.k - и есть коэффициент, вычисленный по формуле Стерджеса):\\
		\texttt{if (y.k > round(y.k)) y.k <- round(y.k) + 1 else y.k <- round(y.k)}\\





	\item Вычислите относительные частоты попадания в каждый интервал разбиения.

	\item Подсчитайте кумулятивные (интегральные) частоты.

	\item Вычислите относительные кумулятивные частоты, которые равны кумулятивным частотам, деленным на число элементов выборки.

\end{enumerate}

\psection{Лабораторная работа №2}

\textbf{\large Корреляция. Статистическое оценивание.}

\textbf{Корреляция.}

\begin{enumerate}

    \item Смоделируйте 2 выборки из 100 значений нормально распределенной случайной величины. \\
          Для этого получите 100 случайных чисел от 0 до 1 с помощью функции \texttt{runif}.
          Затем, при помощи функции \texttt{qnorm}, которая первым аргументом принимает 
          набор [предварительно сгенерированных] вероятностей, необходимо получить две искомых выборки
          по одному и тому же набору вероятностей с различными значениями матожидания и среднего 
          квадратического отклонения.

          \begin{mdframed}[style=BadassFrame]

              \texttt{runif(num, low, high)} -- генерирует \texttt{num} случайных чисел в заданном интервале $[low; high]$  
              \begin{description}

                \item[num] -- количество чисел
                \item[low] -- нижняя граница интервала
                \item[high] -- верхняя граница интервала
              \end{description}
          \end{mdframed}

          \begin{mdframed}[style=BadassFrame]

              \texttt{qnorm(p, mean = 0, sd = 1, lower.tail = TRUE, log.p = FALSE)} 
                -- вычисляет обратное значение функции нормального распределения по набору вероятностей
              \begin{description}

                \item[p] -- набор вероятностей
                \item[mean] -- (необязательный) матожидание случайной величины
                \item[sd] -- (необязательный) среднее квадратическое отклонение
                \item[lower.tail] -- (необязательный) если TRUE (по умолчанию), вероятности берутся в интервале $P[X \leq x]$. Иначе, $P[X > x]$ 
                \item[log.p] -- (необязательный) если TRUE, предполагается, что вероятности p даны как log(p)
              \end{description}
          \end{mdframed}

    \item Вычислите выборочный коэффициент корреляции для полученных выборок. Для вычисления 
          воспользуйтесь функцией \texttt{cor}. Достаточно использовать первые два аргумента.

          \begin{mdframed}[style=BadassFrame]

              \texttt{cor(x, y = NULL, use = ``everything'', \\ method = c(``pearson'', ``kendall'', ``spearman'')} -- вычисляет коэффициент корреляции двух наборов величин. 
              \begin{description}

                \item[x] -- первый набор
                \item[y] -- (необязательный) второй набор (аргумент по умолчанию эквивалентен y=x, но более эффективен )
                \item[use] -- (необязательный) метод вычисления коэффициента ковариации в случае отсутствия некоторых значений.
                    ``everything'' (по умолчанию), ``all.obs", ``complete.obs'', ``na.or.complete'', or ``pairwise.complete.obs''
                \item[method] -- (необязательный) способ вычисления коэффициента ковариации. Должен быть один из: 
                    ``pearson'' (по умолчанию), ``kendall'', ``spearman''
              \end{description}
          \end{mdframed}

    \item Вычислите выборочные коэффициенты корреляции для полученных выборок, используя функцию 
          \texttt{cov}. Достаточно использовать первые два аргумента.

          \begin{mdframed}[style=BadassFrame]

              \texttt{cov(x, y = NULL, use = ``everything'', \\ method = c(``pearson'', ``kendall'', ``spearman''))} 
                    -- вычисляет коэффициент ковариации двух наборов величин. 
              \begin{description}

                \item[x] -- первый набор
                \item[y] -- (необязательный) второй набор (аргумент по умолчанию эквивалентен y=x, но более эффективен )
                \item[use] -- (необязательный) метод вычисления коэффициента ковариации в случае отсутствия некоторых значений.
                    ``everything'' (по умолчанию), ``all.obs'', ``complete.obs'', ``na.or.complete'', or ``pairwise.complete.obs''
                \item[method] -- (необязательный) способ вычисления коэффициента ковариации. Должен быть один из: 
                    ``pearson'' (по умолчанию), ``kendall'', ``spearman''
              \end{description}
          \end{mdframed}

    \item Вычислите средние квадратические отклонения выборок с помощью функции \texttt{sd} и сравните
          с исходными значениями.


    \textbf{Точечная и интервальная оценки параметров распределения. Доверительные интервалы.}
    \item Смоделируйте выборку из 100 значений нормально распределенной случайной величины с помощью
          функции \texttt{rnorm}. Сохраните значения выбранных математического ожидания и 
          среднего квадратического отклонения. Также вычислите дисперсию на основании математического 
          ожидания.

    \item Вычислите дисперсию и среднее квадратическое отклонение выборки с помощью 
          функций \texttt{var} и \texttt{sd}. Сравните полученные значения с ранее использованными.

          \begin{mdframed}[style=BadassFrame]

              \texttt{var(x, y = NULL, na.rm = FALSE, use)} -- вычисляет дисперсию набора величин
              \begin{description}

                \item[x] -- первый набор
                \item[y] -- (необязательный) второй набор (аргумент по умолчанию эквивалентен y=x, но более эффективен )
                \item[na.rm] -- (необязательный) должны ли учитываться отсутствующие значения? (нет по умолчанию).
                \item[use] -- (необязательный) метод вычисления коэффициента ковариации в случае отсутствия некоторых значений.
                    ``everything'' (по умолчанию), ``all.obs", ``complete.obs'', ``na.or.complete'', or ``pairwise.complete.obs''
              \end{description}
          \end{mdframed}

    \item Вычислите вероятность попадания случайной величины в интервал $[5;11]$. Для нахождения значений функции
          нормальной случайной величины воспользуйтесь функцией \texttt{pnorm} соответствующим образом: $P\{5<X<11\}=F(11)-F(5)$.

          \begin{mdframed}[style=BadassFrame]

              \texttt{pnorm(q, mean = 0, sd = 1, lower.tail = TRUE, log.p = FALSE)} 
                -- вычисляет результат функции нормального распределения
              \begin{description}

                \item[q] -- аргумент функции нормального распределения
                \item[mean] -- (необязательный) матожидание случайной величины
                \item[sd] -- (необязательный) среднее квадратическое отклонение
                \item[lower.tail] -- (необязательный) если TRUE (по умолчанию), вероятности берутся в интервале $P[X \leq x]$. Иначе, $P[X > x]$ 
                \item[log.p] -- (необязательный) если TRUE, предполагается, что вероятности p даны как log(p)
              \end{description}
          \end{mdframed}

    \item Получите случайную выборку на 10 элементов из вышеуказанной генеральной совокупности при помощи функции
          \texttt{sample}

          \begin{mdframed}[style=BadassFrame]

              \texttt{sample(x, size, replace = FALSE, prob = NULL)} 
                -- позволяет получить \texttt{size} случайных значений из набора \texttt{x}
              \begin{description}

                \item[x] -- входной набор значений
                \item[size] -- количество (положительное) случайных значений
                \item[replace] -- (необязательный) слудует ли выбирать с замещением элемента
                \item[prob] -- (необязательный) вектор значений вероятности выборки элементов из входного набора значений
              \end{description}
          \end{mdframed}


\end{enumerate}




\psection{Лабораторная работа №3}
\textbf{Парная линейная регрессия.}

Парную линейную регрессию строят при изучении связи между исследуемым показателем и
объясняющей переменной. При наличии статистически значимой линейной связи парную
линейную регрессию можно применять для прогнозирования показателя и для оценки влияния
возможных изменений фактора на показатель.

Исходные данные (из полученного .xls файла с вариантом) представляют собой двумерную
выборку $(x_i, y_i, i = 1..30)$. По выборке необходимо построить парную линейную
регрессию, проверить её статистическую значимость и построить точечный и интервальный
прогнозы для значения зависимой переменной.

\begin{enumerate}
	\item Для заданных исходных данных постройте поле корелляции -- диаграмму зависимости
        показателя $y$ от фактора $x$. 

    \item Вычислите выборочные характеристики:
    \begin{itemize}
        \item[--] выборочные средние $\overline{x}$ и $\overline{y}$;
        \item[--] выборочные дисперсии $S_x^{2}$ и $S_y^{2}$;
        \item[--] выборочное среднее квадратические отклонения $S_x$ и $S_y$;
        \item[--] выборочный коэффициент корреляции $r_{xy}$.
    \end{itemize}

    \item Вычислите коэффициенты выборочной линейной регрессии.
\end{enumerate}
\psection{Лабораторная работа №3}

\psection{Лабораторная работа №4}
\psection{Лабораторная работа №5}

\textbf{Построение парной линейной регрессии с помощью R.}
\begin{enumerate}

	\item 
\end{enumerate}


\end{document}
