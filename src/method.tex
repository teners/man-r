\documentclass[a4paper,12pt]{article} %размер бумаги устанавливаем А4, шрифт 12пунктов
\usepackage[T2A]{fontenc}
\usepackage[utf8]{inputenc}	%кодировка
\usepackage[english,russian]{babel} %используем русский и английский языки с переносами
\usepackage[colorlinks=true,urlcolor=black,linkcolor=black,filecolor=black,citecolor=black]{hyperref}
\usepackage{lipsum}
\usepackage{fancyhdr}
%% set indent at the beginning of a paragraph
\usepackage{indentfirst} 
\usepackage[titletoc]{appendix}
\usepackage{bookmark}
\usepackage{enumitem}
\usepackage{listings}
%% set urls' colors 
\usepackage{amsmath}

\usepackage{geometry} % Меняем поля страницы
\geometry{a4paper, left=2cm, right=1.5cm, top=1cm, bottom=2cm }

\setlength{\parindent}{1cm}
\setlength{\parskip}{1ex plus 0.5ex minus 0.2ex}

%% macro starts new section on new page
\newcommand{\psection}[1]{\newpage\section{#1}}
%% this one does the same but for unenumerate sections
\newcommand{\unsection}[1]{\newpage\section*{#1}\addcontentsline{toc}{section}{#1}}

\begin{document}

\begin{titlepage}

	\begin{center}
		\sc{МИНИСТЕРСТВО ОБРАЗОВАНИЯ И НАУКИ РОССИЙСКОЙ ФЕДЕРАЦИИ 
		ФЕДЕРАЛЬНОЕ ГОСУДАРСТВЕННОЕ АВТОНОМНОЕ ОБРАЗОВАТЕЛЬНОЕ УЧРЕЖДЕНИЕ
		ВЫСШЕГО ОБРАЗОВАНИЯ
		«САНКТ-ПЕТЕРБУРГСКИЙ ГОСУДАРСТВЕННЫЙ УНИВЕРСИТЕТ
		АЭРОКОСМИЧЕСКОГО ПРИБОРОСТРОЕНИЯ»  }

		\vspace{2cm}
		\sc{Авторы: Д.В. Дулин, С.А. Соколов}\\
		{\footnotesize Под редакцией М.А. Нарбута и М.В. Фаттаховой}
	\end{center} 

	\begin{center}
		\begin{tabular}{p{13cm}}
			\vspace{2cm}\\
			\begin{center}
				\vspace{1cm}
				\huge{Методические указания для выполнения лабораторных работ в среде R по курсу прикладной теории вероятностей.}
			\end{center}
		\end{tabular} 
	\end{center}

	\begin{center}
		\vspace{8cm}
		Санкт-Петербург\\
		2016
	\end{center}

\end{titlepage}


\tableofcontents 
\newpage

\unsection{Введение}
\hyphenation{Com-man-der}
\hyphenation{Comp-re-hen-sive R Ar-chi-ve Net-work} 
R – язык программирования и набор программного обеспечения для статистической обработки 
данных с открытым исходным кодом в рамках проекта GNU. R широко используется как 
инструмент для для статистики и интеллектуального анализа данных и фактически является стандартом
для статистических программ. Распространяется в виде исходных кодов и откомпилированных
приложений под Unix/Linux, MacOS и Windows. \\
\indent В основном в R используется интерфейс командой строки, хотя доступны некоторые графические
интерфейсы, которые можно свободно найти в сети (например, пакет R Commander, который так
же свободно распространяется под лицензией GNU GPL). \\ 
\indent R поддерживает широкий спектр статистических и численных методов и обладает хорошей расширяемостью 
с помощью пакетов. Пакеты представляют собой библиотеки для работы специфических функций или 
специальных областей применения. Полный список доступных пакетов, отсортированных по областям 
применения можно найти на сервере Comprehensive R Archive Network (https://cran.r-project.org/web/views/). Так же, R имеет огромные 
возможности для создания качественной графики, например, диаграммы, графики, математические формулы 
(в этих целях авторы рекомендуют использовать не менее популярную систему TeX). \\
\indent Являясь, в первую очередь, интерпретируемым языком, R позволяет сразу же получать ответ на введённую
команду. К примеру, если пользователь вводит "2+2" в командную строку и нажимает enter, сразу же
отображается ответ, что демонстрируется ниже: \\ \\
\indent \texttt{> 2+2} \\ 
\indent \texttt{[1] 4} \\ \\
\indent Как и большинство подобных языков, таких как APL и MATLAB, R поддерживает матричную арифметику.
Одними из примитивных структур данных в R являются списки, векторы, массивы, матрицы, таблцы.
Примерами сложных структур являются регрессионные модели, временные ряды, пространственные 
(географические) координаты и т. д. Стоит отметить, что в R отсутствуют простые типы данных, как,
например, в C или Java. Вместо этого, любая скалярная величина представляется, как вектор длины один.




\end{document}
