\documentclass[a4paper,12pt]{article} %размер бумаги устанавливаем А4, шрифт 12пунктов
\usepackage[T2A]{fontenc}
\usepackage[utf8]{inputenc}	%кодировка
\usepackage[english,russian]{babel} %используем русский и английский языки с переносами
\usepackage[colorlinks=true,urlcolor=black,linkcolor=black,filecolor=black,citecolor=black]{hyperref}
\usepackage{lipsum}
\usepackage{fancyhdr}
%% set indent at the beginning of a paragraph
\usepackage{indentfirst} 
\usepackage[titletoc]{appendix}
\usepackage{bookmark}
\usepackage{enumitem}
\usepackage{listings}
%% set urls' colors 
\usepackage{amsmath}

\usepackage{geometry} % Меняем поля страницы
\geometry{a4paper, left=2cm, right=1.5cm, top=1cm, bottom=2cm }

\setlength{\parindent}{1cm}
\setlength{\parskip}{1ex plus 0.5ex minus 0.2ex}

%% macro starts new section on new page
\newcommand{\psection}[1]{\newpage\section{#1}}
%% this one does the same but for unenumerate sections
\newcommand{\unsection}[1]{\newpage\section*{#1}\addcontentsline{toc}{section}{#1}}

\begin{document}

\begin{titlepage}

	\begin{center}
		\sc{МИНИСТЕРСТВО ОБРАЗОВАНИЯ И НАУКИ РОССИЙСКОЙ ФЕДЕРАЦИИ 
		ФЕДЕРАЛЬНОЕ ГОСУДАРСТВЕННОЕ АВТОНОМНОЕ ОБРАЗОВАТЕЛЬНОЕ УЧРЕЖДЕНИЕ
		ВЫСШЕГО ОБРАЗОВАНИЯ
		«САНКТ-ПЕТЕРБУРГСКИЙ ГОСУДАРСТВЕННЫЙ УНИВЕРСИТЕТ
		АЭРОКОСМИЧЕСКОГО ПРИБОРОСТРОЕНИЯ»  }

		\vspace{2cm}
		\sc{Авторы: Д.В. Дулин, С.А. Соколов}\\
		{\footnotesize Под редакцией М.А. Нарбута и М.В. Фаттаховой}
	\end{center} 

	\begin{center}
		\begin{tabular}{p{13cm}}
			\vspace{2cm}\\
			\begin{center}
				\vspace{1cm}
				\huge{Методические указания для выполнения лабораторных работ в среде R по курсу прикладной теории вероятностей.}
			\end{center}
		\end{tabular} 
	\end{center}

	\begin{center}
		\vspace{8cm}
		Санкт-Петербург\\
		2016
	\end{center}

\end{titlepage}


\tableofcontents 
\newpage

\unsection{Введение}
    \hyphenation{Com-man-der}
    \hyphenation{Comp-re-hen-sive R Ar-chi-ve Net-work} 

    R – язык программирования и набор программного обеспечения для статистической обработки 
    данных с открытым исходным кодом в рамках проекта GNU. R широко используется как 
    инструмент для для статистики и интеллектуального анализа данных и фактически является стандартом
    для статистических программ. Распространяется в виде исходных кодов и откомпилированных
    приложений под Unix/Linux, MacOS и Windows. 
    
    \indent В основном в R используется интерфейс командой строки, хотя доступны некоторые графические
    интерфейсы, которые можно свободно найти в сети (например, пакет R Commander, который так
    же свободно распространяется под лицензией GNU GPL).  
    
    \indent R поддерживает широкий спектр статистических и численных методов и обладает хорошей расширяемостью 
    с помощью пакетов. Пакеты представляют собой библиотеки для работы специфических функций или 
    специальных областей применения. Полный список доступных пакетов, отсортированных по областям 
    применения можно найти на сервере Comprehensive R Archive Network (\textbf{https://cran.r-project.org/web/views/}). Так же, R имеет огромные 
    возможности для создания качественной графики, например, диаграммы, графики, математические формулы 
    (в этих целях авторы рекомендуют использовать не менее популярную систему TeX). 
    
    \indent R является интерпретируемым языком, что позволяет сразу же получать ответ на введённую
    команду. К примеру, если пользователь вводит \texttt{2+2} в командную строку и нажимает enter, сразу же
    отображается ответ, что демонстрируется ниже: \\
    \indent \texttt{> 2+2} \\ 
    \indent \texttt{[1] 4} 
    
    \indent Как и большинство подобных языков, таких как APL и GNU Octave, R поддерживает матричную арифметику.
    Одними из примитивных структур данных в R являются списки, векторы, массивы, матрицы, таблцы.
    Примерами сложных структур являются регрессионные модели, временные ряды, пространственные 
    (географические) координаты и т. д. Стоит отметить, что в R отсутствуют простые типы данных, как,
    например, в C или Java. Вместо этого, любая скалярная величина представляется, как вектор длины один.  
    
    \indent Документацию по R на русском языке можно найти здесь: \url{http://herba.msu.ru/shipunov/software/r/r-ru.htm}. \\
    \indent И на английском языке здесь: \url{https://stat.ethz.ch/R-manual/R-devel/doc/html/}.

\psection {Базовый синтаксис R}
    В данной главе продемонстрирован основной синтаксис и использование R в интерфейсе 
    командной строки. Обычно, пользовательская сессия начинается с ввода \texttt{R} в 
    терминале, либо с запуска приложения с графическим интерфейсом. После этого можно вводить
    команды как в обычной командной строке.\\

    \indent R -- объектно-ориентированный язык программирования. Это значит, что почти всё что 
    угодно можно хранить в R как объект. Каждый объект имеет свой класс. Класс описывает
    содержимое объекта и то, что каждая из функций может с ним делать. Например, функция
    \texttt{plot(x)} производит различные графики, в зависимости от того, является ли 
    \texttt{x}, например, вектором или регрессионной моделью. \\
    
    \indent Оператор присвоения в R выглядит как \texttt{<-}, хотя классический оператор \texttt{=}
    также может быть использован. Две следующие записи эквивалентны: \\
    \indent \texttt{> x <- 2} \\
    \indent \texttt{> x = 2} \\ \\
    
    \indent Аргументы в функции передаются внутри круглых скобок. Таким способом можно использовать 
    результат выполнения функции как аргумент другой функции: \\
    \indent \texttt{mean(rnorm(10)\^{}2)} \\ \\
    
    \indent C символа \texttt{\#} начинается комментарий до конца строки: \\
    \indent \texttt{\# Это комментарий} \\
    \indent \texttt{2+2 \# Это тоже комментарий} \\ \\
    
    \indent R чувствителен к регистру. \texttt{X} и \texttt{x} -- это разные объекты. \\
    
    \indent Рассмотрим некоторые операции присвоения и применения стандартных функций. Как мы
    знаем, объекты могут хранить скалярные и строковые данные: \\
    \indent \texttt{> x <- 2} \\
    \indent \texttt{> x} \\
    \indent \texttt{[1] 2} \\
    \indent \texttt{> x <- "foo"} \\
    \indent \texttt{> x } \\
    \indent \texttt{[1] "foo"} \\ \\
    
    \indent Аналогично, объекты могут хранить векторы: \\
    \indent \texttt{> foo <- c(1, 2, 3, 4, 5)} \\
    \indent \texttt{> foo} \\
    \indent \texttt{[1] 1, 2, 3, 4, 5} \\
    \indent \texttt{> bar <- c(6, 7, 8, 9, 10)} \\
    \indent \texttt{> c(foo, bar) \# функция c() - конкатенация объектов} \\
    \indent \texttt{[1] 1, 2, 3, 4, 5, 6, 7, 8, 9, 10} \\ \\

    \indent Из этой главы можно сделать вывод, что R так же прост в обращении, как, например, Python. 
    Математические и статистические функции языка R, необходимые для выполнения лабораторных 
    работ, будут описаны в соответствующих главах.
    


\end{document}
