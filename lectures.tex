% vim: set fenc=utf-8 ts=2 sw=2 sts=2 tw=79 cc=80 noet foldmethod=indent:

\documentclass[a4paper,12pt]{report}
\usepackage[utf8]{inputenc}
\usepackage[OT2,T2A,T1]{fontenc}
\usepackage[english,russian]{babel}
\usepackage{enumitem}
\usepackage{array}
\usepackage[pdfauthor={Danil Dulin},
            pdftitle={Lectures: AI Systems},
            colorlinks=true,linkcolor=black]{hyperref}
\usepackage{listings}
\usepackage[dvipsnames]{xcolor}
\usepackage{graphicx}
\usepackage{indentfirst}
\usepackage{mathtools}
\usepackage{mathrsfs}
\usepackage{tikz}

\usepackage{fontspec}
\setmainfont{Liberation Serif}

\usepackage{geometry}
\geometry{a4paper, left=1cm, right=1cm, top=2cm, bottom=2cm}

\setlist{nolistsep}
\def\arraystretch{1.3}
\setlength\parindent{24pt}

\begin{document}
\author{Danil Dulin}
\title{Lectures: AI Systems}

{\huge{Системы Искусственного Интеллекта}}

\tableofcontents

\newpage


\chapter{Унифицированная информационная технология проектирования и
	экспулатации программных комплексов автоматизированного мониторинга
	состояния и управления сложными техническими объектами}


\section{Техническая актуальность и экономическая целесообразность
	разработки новой информационной технологии разработки экспулатации
	программных комплексов автоматизированного мониторинга состояния и
	управления сложными техническими объектами}

	Глобальный характер и сложность задач возлагаемых на космические средства
	(средства наведения, орбитальные средства, средства наземного комплекса
	управления) обусловили резкое возрастание объемов информации, поступающей
	в центры управления полетами, командные пункты, усиление ``жесткости``
	 временных и других ресурсных ограничений, при принятии решений с
	одновременным сокращением финансовых средств. Для того, чтобы эффективность
	применения космических средств не уменьшалась, а возрастала в таких
	условиях, необходимо наличие соответствующих информационных технологий
	и реализующих их программных средств. Для реализации названной
	информационной технологии в настоящее время слабо решены вопросы:
	\begin{itemize}
		\item накопления знаний;
		\item организации распределенных вычислений;
		\item создания (синтеза) соответствующих математичесикх моделей;
		\item организации хранения и доступа к данным и знаниям;
		\item проектирования и реализации интеллектуального интерфейса;
		\item борьбы с искажениями и надежности вычислений, и т.д.
	\end{itemize}


\section{Текущее состояние в области проектирования и эксплуатации
	программных средств, мониторинг состояния и управления}

	Современное состояние проблемы автоматизации управления сложными техническими
	объектами можно охарактеризовать как переход от стихийного этапа, когда
	применяется метод ``проб и ошибок`` при проектировании, к сознательному,
	характерной чертой которого является обязательное обоснование применяемых
	математических моделей и методов.\\

	\begin{itemize}
		\item БММ — Банк Математических Моделей.
		\item САПР — Система Автоматического Проектирования.
		\item БЗ — База Знаний. \\
	\end{itemize}

	Из БММ данные поступают в САПР для получения на выходе готового продукта.\\

	Системы Поддержки Принятия Решений:
	\begin{enumerate}
		\item данные;
		\item информация;
		\item знания;
		\item решения.
	\end{enumerate}


\section{Характеристика основных элементов информационной технологии
	информационных технологий мониторингового состояния и управления}


\subsection{Назначение информационной технологии}

	Основное назначение информационной технологии мониторга состояния и
	управления проведение мониторинга состояния объекта управления и формирование
	выдачи управляющих воздействий, осуществляемое в условиях различных
	ограничений.

	Мониторинг состояния предполагает получение в явном виде обобщенных оценок
	выполнения программного функционирования объекта управления, либо степени
	работоспособности, либо места и вида возникшей неисправности, либо оценок
	прогнозируемых явлений и процессов — с учетом конкретных целей и условий
	применений рассматриваемого объекта управления.

	Формирование и выдача управляющих воздействий предполагает синтез программны
	применения конкретного объекта управления, выработка соответсвующих
	управляющих воздействий, доведение их до объекта управления и получение
	суждения о реализации данного управляющего объекта. Как правило, формирование
	таких урпавляющих воздействий производится на основе модели оптимального,
	квазиоптимального или сатисфакицонного выбора.

	Задачи систем мониторинга состояния и управления:
	\begin{enumerate}
		\item контроль функционирования;
		\item контроль работоспособности;
		\item прогнозирование поведения;
		\item формирование и выдача управляющих последовательностей.
	\end{enumerate}


\subsection{Характеристика унифицированной программной платформы в рамках
	информационной технологии}

	Автоматизация невозомжна без формализации данных и знаний в которых
	нуждается система.


\subsection{Структура операционной системы}

	Куб

	\begin{enumerate}
		\item Языковые средства
			\begin{enumerate}
				\item Подсистема описания измеряемых и вычисляемых параметров.
					Их характеристики, откуда приходят, мощность и т.д.
				\item Подсистема групп параметров.
				\item Подсистема описания моделей сегментации значений параметров.
				\item Подсистема описания динамических моделей значений парметров.
				\item Подсистема описания диалоговых панелей.
				\item Подсистема многофункциональных запросов к базе данных.
				\item Специализированная подсистема контроля функционирлования
					(диагностирования).
				\item Подсистема проверки условий и изменения хода вычислительного
					процесса.
				\item Подсистема подсистема вызова автономных операций над данными.
				\item Подсистема графического/мультимедийного отображения.
				\item Подсистема организации диалога с конечным пользователем -
					экспертом.
				\item Подсистема организации контроля динамических процессов -
					циклограмм.
			\end{enumerate}
		\item Система баз данных
		\item Специализированные интерактивные системы
			\begin{enumerate}
				\item Подсистема концептуального можделирования связанная со знанием.
				\item Подсистема поведенческого моделирования.
				\item Подсистема генерации графического пользовательского интерфейса.
				\item Подсистема автоматического синтеза коректной и оптимальной
					метапрограммы.
			\end{enumerate}
	\end{enumerate}

	Вид сбоку, содержимое [1] и [2]:

	\begin{itemize}
		\item 2/3 - ОС
			\begin{itemize}
				\item содержит элемент компонентных вычислений;
				\item содержит сетевые средства.
			\end{itemize}
		\item 1/3 - Монитор ОС.
	\end{itemize}


\chapter{Системы искусственного интеллекта и их использование при построении
	автоматизированных средств и систем мониторинга состояния и управления}

Список литературы:

\begin{enumerate}
	\item Ефимов Е.И. — Решатели интеллектуальных задач;
	\item Кузин Л.Т. — Основы кибернетики: в двух домах:
		\begin{itemize}
			\item Т.1. Математические основы кибернетики;
			\item Т.2. Основы кибернетических моделей.
		\end{itemize}
	\item Нильсон Н. — Приниципы искусственного интеллекта;
	\item Симонс Дж. — ЭВМ пятого поколения;
	\item Эндрю А. — Искусственный интеллект;
	\item Логический подход к искусственному интеллекту: от классической логики
		к логическому программированию.
\end{enumerate}


\section{Этапы развития систем искусственного интеллекта}

\begin{tabular}{ c | c | c | c }
	1950 & 1970 & 1980 & 1990-2010 \\
	\hline
	& I Этап & II Этап & III Этап
\end{tabular}


\subsection{Характеристика I этапа}
	Разработка и создание систем искусственного интеллекта
	решающих задачи на основе применения эвристических методов. \par
	Эвристический метод - свойственный человеческому мышлению метод, для которого
	характерно использование догадок о путях решения задачи с последующей их
	проверкой. \par
	Алгоритм - строгое и точное предписание и выполнение. Полная
	противоположность эвристике. \par
	Первоначально, широкое распространение получили разработки,
	моделирующие мыслительную деятельность человека.\par

	Область применения на I этапе:
	\begin{itemize}
		\item игры;
		\item головоломки;
		\item математические задачи.\\
	\end{itemize}

	Все они характеризовались простотой, ясностью и очевидностью предметной
	области (проблемной среды), ее относительно малой громоздкостью и прочее. \\

	Основная характеристика I этапа - искуссственные предметные области.


\subsection{Характеристика II этапа}
	Разработка и создание систем искусственного интеллекта,
	ориентированных на применение интегральных роботов. \par
	При переходе на первом этапе от искусственных предметных областей к реальным
	предметным областям, разработчики стали наталкиваться на большие трудности,
	обусловленные необходимостью можелирования реального мира:
	\begin{itemize}
		\item необходимость описания знаний о внешнем мире;
		\item необходимость хранения знаний о внешнем мире;
		\item необходимость эффективного поиска и доступа к знаниям;
		\item необходимость проверки знаний на корректность,
			непротиворечивость, полноту.\\
	\end{itemize}

	Практическое применние систем искусственного интеллекта на втором этапе
	привело к созданию интегральных роботов, которые, в свою очередь, привели к
	необходимости рассмотрения и использования релальных предметных областей.\par
	Переход к реальным предметным областям обусловил переход от ислледований по
	моделированию способов мысления человека к разрабоке программ, способных
	решать ``человеческие задачи``, но формальными (математическими)
	методами.\par
	Для систем искусственного интеллекта II этапа характерно:
	\begin{itemize}
		\item модель предметной области;
		\item алгоритмы распознавания ситуации или образов, анализа изображений и
			сцен (Теория распознавания образов);
		\item алгоритм принятия решений;
		\item алгоритм планирования интегрального робота, составление плана и его
			отслеживание;
		\item алгоритмы оценки выполнения качества работы.\\
	\end{itemize}

	Появление первых образцов интегральных роботов, решающих свои задачи в
	комплексе, показало необходимость решения кординальных проблем,
	связанных со следующими задачами:
	\begin{itemize}
		\item представление знаний;
		\item зрительное восприятие (Распознавание образов);
		\item построение сложных планов работы интегральных роботов;
		\item общение с роботами на естественном языке.
	\end{itemize}


\subsection{Характеристика III этапа}
	Разработка и создание систем искусственного интеллекта,
	характерной чертой которых явился переход от создания автономно
	функционирующих систем, самостоятельно решающих задачи в реальной среде,
	к созданию человеко-машинных систем, совмещающих (комбинирующих) достоинства
	интеллекта человека и возможности ЭВМ — для достижения общей для них цели —
	решение задачи, поставленной перед интегральной человеко-машинной
	системой.\par

	Причины возникновения человеко-машинных систем искусственного интеллекта:

	\begin{enumerate}
		\item К этому времени выяснилось, что даже простые задачи функционирования
			интегрального робота в реальных предметных областях не могут быть решены
			методами, разработанными для роботов в эксперементальных
			предметных областях;
		\item Стало также ясно, что сочетание дополняющих друг друга способностей
			человека (эвристика, интуиция, опыт) и возможностей компьютера
			(быстродействие, большие объемы памяти, надежность, живучесть) позволяет
			обойти острые углы путем перекладывания на человека тех функций,
			которые пока еще не доступны для компьютера. \\
	\end{enumerate}

	Таким образом под СИИ понимается такая система, в которой:
	\begin{enumerate}
		\item развиваются возможности компьютера в направлении обеспечения
			совместного с пользователем решения задач;
		\item упрощаются процессы общения человека с компьютером в ходе решения
			задач;
		\item постоянно расширяется доля компьютеров в совместной деятельности
			с человеком по решению задач;
		\item значительное внимание уделяется повышению способности компьютера
			к самомстоятельному решению (в автоматическом режиме) трудно решаемых
			задач.
	\end{enumerate}


\section{Новая информационная система обработки информации}


\subsection{Основные черты традиционной обработки информации}

	Наличие следующих этапов:
	\begin{itemize}
		\item
			\begin{tabular}{ p{2cm} | c | p{2cm} | c | p{2.4cm} | c | p{2cm} }
				& Задача        & & Задача        & & Реализация    & \\
				& $\rightarrow$ & & $\rightarrow$ & & $\rightarrow$ & \\
				\hline
				Эксперт & & Системный аналитик & & Программист & & Компьютер \\
				\hline
				& $\leftarrow$ & & $\leftarrow$ & & $\leftarrow$ & \\
				& Формулировка & & Постановка   & & Реализация   & \\
			\end{tabular}
		\item Большая трудоёмкость решения конкретной задачи обработки информации,
			связанная с имеющимися место на каждом этапе:
			\begin{itemize}
				\item ошибками;
				\item неточностями;
				\item нерациональными решениями;
				\item необходимостью внесения изменений в схемы решения,
					методы решения, постановку задачи, программу.
			\end{itemize}
		\item Сложность сопровождения разработанного программного обеспечения.
		\item ПО в традиционной технологии основывается на формальной
			(математической) модели задач и представления данных, в то время как
			каждая конкретная предметная область основывается на системе,
			содержательных понятий, которыми оперирует пользователь при решении
			задач. Это значит, что в такой схеме решения система понятий предметной
			области и формальной модели не совпадают.
		\item При формулировке задачи конечный пользователь должен перевести
			постановку задачи, выраженной в системе предметной области в постановку
			задачи формальной модели, при получении результата — наоборот.
	\end{itemize}


\subsection{Основная идея новой технологии обработки данных}

	Система понятий конкретной предметной области рассматривается как исходная
	информация для решения прикладных задач, при этом обеспечивается
	автоматическая интерпретация системы понятий формальной модели.\par
	Для новой технологии характерна следующая цепочка обработки данных:
	\begin{enumerate}
		\item Специалист (эксперт) — конечный пользователь;
		\item Содержательная постановка задачи;
		\item Вычислительная система.
	\end{enumerate}
	В новой технологии обработки информации вычислительной системе должна
	задаваться только постановка задачи в виде описания требуемого результата и
	условий его получения, в то время как последовательность операций,
	посредством которых она решается, определяется решающей задачу системой.\par
	Вычислительная система в составе человеко-машинной системы превращается из
	пассивного звена в активную целенаправленную систему, целью которой является
	получение требуемого в постановке задачи результата или знания.\par
	Получение этого знания достигается:
	\begin{itemize}
		\item автоматическим синтезом (генерацией) вычислительной системы;
		\item выполнением оптимальной в определённом смысле последовательности
			вычислительных, логических и поисковых операций над имеющимися знаниями.
	\end{itemize}


\subsection{Структура информационной вычислительной системы в новой
	технологии обработки информации}

	Вычислительная система:
	\begin{enumerate}
		\item Модель языка пользователя.
		\item Модель представления знаний.
		\item База знаний.
		\item Интеллектуальный интерфейс:
			\begin{itemize}
				\item система общения;
				\item решатель.
			\end{itemize}
		\item Исполнительная система.
	\end{enumerate}

	Интеллектуальным интерфейсом - система программных и аппаратных средств,
	обеспечивающих для конечного пользователя, не имеющего специальной подготовки
	в области вычислительной системы для решения задач, возникающих в сфере
	профессиональной деятельности либо без посредников-программистов либо с
	незначительной их помощью. \par
	Процесс внедрения средств интеллектуального интерфейса в вычислительную
	технику определяется как интеллектуализация. \par
	Система общения — совокупность средств, осуществляющих трансляцию с языка
	пользователя на язык представления знаний в базе знаний и включающую
	в себя средства трансляции и средства, обеспечивающие понимание. \par
	Решатель — совокупность средств, обеспечивающих в диалоге с пользователем
	автоматический синтез программы решения задач, т.е. анализ условий задачи,
	выделение подзадач, имеющих стандартные решения, и объединение этих подзадач
	в единую целостную программу. Функционирование системы, имеющей в своем
	составе решатель, может осуществляться в 2-х режимах: в режиме компиляции
	и режиме интерпретации. \par
	Исполнительная система представляет собой совокупность средств,
	выполняющих синтезированную программу вычислений, сформированную с позиции
	эффективного решения задач и имеет проблемную ориентацию. \par
	База знаний занимает центральное положение по отношению к остальным
	компонентам вычислительной системы. Через базу знаний осуществляется
	интеграция всех средств вычислительной системы, участвующих в решении
	конкретных задач. \par
	Базу знаний для вычислительных систем, ориентированных на использование
	новой технологии обработки данных целесообразно строить как 2-уровневую
	структуру, включающую 2 компонента:
	\begin{enumerate}
		\item концептуальную базу знаний (верхний концептуальный уровень);
		\item базы данных (нижний информационный уровень).
	\end{enumerate}

	Такое деление позволяет обеспечить эффективное представление обобщенных
	знаний и метазнаний на верхнем концептуальном уровне и конкретной информации
	на нижнем уровне. \par
	Использование баз данных в качестве отдельного компонента баз знаний
	позволяет использовать современные СУБД для описания взаимодействий между
	объектами проблемной среды, причем на уровне конкретных фактов. \par
	Язык представления знаний — это конкретный способ описания знаний в
	проблемной среде, задаваемый синтаксисом описания и правилами соотнесения
	языковых выражений с конкретными объектами проблемной области
	или интерпретацией.


\section{Интеллектуальные информационные технологии}

	За последние годы можно часто услышать тезис о том, что интеллектуальные
	системы обработки информации достигли своего предела, но это совсем не
	так.\par
	Для оценки в области интеллектуальных информационных технологий необходимо
	рассматривать три направления, говорящих об обратном:
	\begin{enumerate}
		\item развитие аппарата знаний;
		\item влияние этого развития на интеллектуализацию ЭВМ;
		\item появление нового поколения интеллектуальных информационных
			технологий, как результат вышеописанного.
	\end{enumerate}


\subsection{Аппарат знаний}

	В настоящее время аппарат знаний предполагает три направления:
	\begin{itemize}
		\item группа декларативных моделей представления знаний;
		\item группа процедурных моделей представления знаний;
		\item группа специальных/комбинированных моделей представления знаний. \\
	\end{itemize}

	Основные направления развития аппарата знаний:
	\begin{itemize}
		\item Извлечение знаний из различных источников
		\item Приобретение знаний от профессионалов
		\item Представление знаний
			\begin{itemize}
				\item Модели знаний
					\begin{itemize}
						\item Семантические сети
						\item Сети фреймов
						\item Логические сети
						\item Продукционные сети
					\end{itemize}
				\item Системы представления знаний
				\item Базы знаний
			\end{itemize}
		\item Манипулирование данными
		\item Объяснение знаний \\
	\end{itemize}

	Типы моделей представления знаний:
	\begin{itemize}
		\item декларативные
			\begin{itemize}
				\item продукционные;
				\item редукционные;
				\item предикатные.
			\end{itemize}
		\item процедурные
			\begin{itemize}
				\item PLANNER;
				\item CONNIVER;
				\item ПРИЗ.
			\end{itemize}
		\item специальные/комбинированные
			\begin{itemize}
				\item семантические сети;
				\item сети фреймов;
				\item нечёткие модели представления знаний;
				\item реляционные модели представления знаний.
			\end{itemize}
	\end{itemize}


\subsection{Интеллектуализация информационных технологий в плане аппарата
	знаний}

	Развитие аппарата знаний оказывает постоянное влияние на развитие
	информационных технологий.

	\begin{itemize}
		\item Переход от классических вычислений к альтернативным методам
			вычислений. \\
			Алгоритм с самого начала своего использования был основой
			Фон-Нейманновской архитектуры вычислений. Однако в течение последних
			десятилетий велись разработки альтернативных способов реализации
			информационных процессов с использованием иссскуственного интеллекта и
			паралелльного программирования для многопроцессорных систем. Качественный
			прогресс в решении этой проблемы обеспечил новый аппарат на основе
			недоопределенных вычислений (программирование в ограничениях), по скольку
			все эти подходы строятся на:
			\begin{itemize}
				\item децентраллизованности;
				\item асинхронности;
				\item параллелизме;
				\item управлении по данным.
			\end{itemize}
		\item Технология активных объектов. \\
			Ключевым в перестройке всех информационных технологий за последние
			десятилетия стало развитие объектно-ориентированного программирования.
			Однако этот подход определяет лишь фундамент будещего IT, оставляя
			прежний алгоритмический характер управления вычислительным процессом. Тем
			временем развитие управления данными и дальнешее управления формирует
			следующее поколение информационных обхектов.
		\item Приоритет моделей, а не алгоритмов.
		\item Паралеллизм. \\
	\end{itemize}

	Нерешеннсоть проблемы распараллеливания императивных вычислений уже несколько
	десятилетий создает непреодолимый барьер на пути распространения
	многопроцессорных систем. За это время стоимость разработки программной и
	проектирования аппаратной части позволяет производить компьютеры с любым
	количество процессоров, однако адаптация существеющих и создание
	новых программных средств остается основной проблемой.\par
	Все эти и некоторые другие тенденции развития информационных технологий
	приведут со временем к тому, что прикладные программные системы будут
	строиться на приоритетности моделей, мультиагентстве и ассоциативном
	самоорганизующемся децентрализованном параллельном вычислительном процессоре.


\subsection{Новое поколение приложений в интеллектуальных информационных
	системах}

	Прикладные программные системы спроектированны с учетом вышеназванных
	тенденций радикально отличаются от традиционных программных систем.\par
	Среди этих новых приложений можно выделить:
	\begin{enumerate}
		\item Интеллектуальные системы реального времени.\\
			Ограничение на получение результатов в реальном времени побуждает
			использовать такие модели представления знаний, которые позволяют
			организовывать потоковые вычисления ориентированные на управление по
			данным. Такие информационные технологии, на которых базируются системы
			реального времени должны иметь развитый аппарат сбора или формирования
			знаний, извлечения этих изнаний из экспертов, пополнения и верификации
			этих знаний, кроме того, должна быть система объяснения этих знаний.
			Такие системы должны быть существенно открытыми и к ним применим принцип
			недоопределенности, когда в рекурсивном режиме происходит доопределение и
			уточнение базы знаний.
		\item Активные объектно-ориентированные СУБД.\\
			Переход от реляционных СУБД к объектно-ориентированным существенно
			запаздывает. Это связанно с инерцией эволюции больших баз данных.
			Внедрение управления по данным позволяет превратить современные СУБД в
			интеллектуальные объектно-ориентированные системы. Мощный виртуальный
			процессор может обеспечить пользователю взаимодействие со сложными
			структурами данных объединяющих сотни или тысячи таблиц, реализующих
			вычисления проверки целостности и точности информации, возможность
			использованя реальных, неточных данных.
		\item Системы автоматического проектирования и Автоматизированные системы
			управления.
		\item Естественный язык и голос.
	\end{enumerate}


\section{Смена парадигмы разработки прикладных систем искусственного
	интеллекта}

	Известен тезис Никлауса Вирта: "Алгоритмы + Структуры Данных = Программа".
	Эти слова являеются отражением того, что в теории программирования имеется
	два противоположных и взаимодополняющих друг друга понятия: одно из них -
	императивное (алгоритмическое, командное), другое - декларативное
	(описательное, непроцедурное). По изменению расстановки акцентов на эти два
	понятия приоритетность их применения и учета их в теории и практике
	программирования можно проследить всю историю развития информационных
	технологий.


\subsection{История развития информационных технологий}

	Под моделью можно понимать определенное множество абстрактных объектов или
	несколько множеств абстрактных объектов различной природы, различающихся по
	условно припысываемым им именам — в совокупности заданной системой отношений
	между элементами этих множеств.\par
	Под алгоритмом можно понимать точные предписания, некоторые
	последовательности действий, необходимые для получения заданного или
	исходного результата.\par
	В настоящее время считается, что модель в большинстве случаев используется
	лишь для формального описания объекта исследования(вычисления), определяя что
	необходимо вычислить, а алгоритм, в сво очередь, являясь основой
	вычислительного процесса определяет ответ на вопрос: "Как нужно
	вычислять?".\par
	Из этих определений ясно, что без установления того, \textbf{что} надо
	вычислить процесс вычислений смысла не имеет, поэтому без можели обойтись
	невозможно. С другой стороны ясно и то, что без алгоритма, определяющего
	\textbf{как} нужно проводить вычисления, тоже не обойтись.\par
	Однако базируюясь на методологии теории искусственного интеллекта, алгоритм,
	как наиболее трудный для понимания конечного пользования элемент, можно
	исключить из спецификации задачи.\par

	Этапы развития информационных технологий:
	\begin{itemize}
		\item Этап I \\ \\
			\begin{tabular}{ c | c | c | c }
				\hline
				Модели & \multicolumn{3}{| c }{Алгоритмы} \\
				\hline
				& Раздел данных & Оператор обработки данных & Оператор управления \\
				\hline
			\end{tabular} \\
		\item Этап II \\ \\
			\begin{tabular}{ c | c | c | c }
				\hline
				\multicolumn{2}{ c }{Модели} & \multicolumn{2}{| c }{Алгоритмы} \\
				\hline
				Данные & Объекты данных & Операторы обработки данных & Операторы
				управления \\
				\hline
			\end{tabular} \\
		\item Этап III \\ \\
			\begin{tabular}{ c | c | c | c }
				\hline
				\multicolumn{3}{ c |}{Модели} & Алгоритмы \\
				\hline
				Данные & Модели данных & Модели знаний  &  Решатель \\
				\hline
			\end{tabular} \\
	\end{itemize}

	Таким образом, если на первом этапе развития информационных технологий раздел
	данных можно было рассматривать как неразвитый элемнт в большом тедле
	алгоритма, то интеллектуальный решатель на третьем этапе представляет собой
	небобльшой автономный элемент алгоритма в теле моделей.


\subsection{Смена влияния моделей и алгоритмов на развитие информационных
	технологий}

	В подтверждение сказанному можно привести следующу аналогию: на заре развития
	вычислительной техники программирование в кодах считалось единственным
	способом обзения с ЭВМ, однако к настоящему времени этот тип программирования
	сохранил свое место в довольно тонком, ближайшем к аппаратной части, слое
	операционных систем, так и модель завоевывает свое место в практике
	программирования. Именно модель представляет собой объект исследования и
	определяет характер формального аппарата, используемого для описания той или
	иной задачи. С моделью работает конечный пользователь - специалист или
	эксперт в конкретной предметной области, с алгоритмом же работает только
	профессиональный программист и разработчик вычислительных методов. Все это
	определило парадоксальность современоого состояния в прикладных
	математических технологиях, когда модель можно встретить в основном только в
	теоретических работах, в качестве иллюстрации к объекту исследований.

	Некоторые свойства сопоставления понятий модели и алгоритма

	\begin{tabular}{ p{0.8cm} | p{7.5cm} | p{8.5cm} }
		\hline
		№ & Модель & Алгоритм \\
		\hline
		1 & Принципиально декларативна & Недекларативен \\
		\hline
		2 & Симметрична по отношению к параметрам, по скольку все они неявным
		образом определяются одни через другие
		& Разделяет параметры на входные и выходные, явным образом определяя вторые
		через первые \\
		\hline
		3 & В неявной форме определяет решения \textbf{всех} задач, связанных с
		этой моделью
		& Определеяет в явной форме и задает решение только \textbf{одной} задаче,
		отношение которой к реальному объекту исследований не всегда очевидно \\
		\hline
		4 & Модель может быть недоопределенной
		& Алгоритм и недоопределенность - несовместимые понятия \\
		\hline
		5 & В общем случае определеяет все пространство решений
		& Традиционный алгоритм позволяет получать только отдельные точечные
		решения \\
		\hline
	\end{tabular} \\

	Подытожив сказаное, можно отметить, что на самом деле модель и возможность
	прямого взаимодействия с ней являлись с самого начала ключевым ориентиром
	исследований в области систем искусственного интеллекта и естественным
	следствием этих исследований была принципиальная потребность выхода за
	пределы парадигмы алгоритма в самых различных приложениях: Lisp, Prolog,
	Фреймы, Мультиагентные системы. Прогноз предсказывает, что через 10-15 лет
	алгоритм ожидает судьба ассемблера и программирования в кодах. Потеря
	сегодняшних ключевых позиций в сравнительно тонком базовом уровне будущего
	информационных технологий.


\section{Задача обработки и анализа измерительной информации, как задача
	распознавания образов}


\subsection{Вводные определения}

	Техническим состоянием объекта контроля (управления) будем называть
	совокупность изменяющихся в процессе производства, испытаний, эксплуатации
	свойств или качеств этого объекта контроля (управления), характеризующих его
	функциональную пригодность в заданных условаиях применения.\par
	Техническое состояние определяется путем оценивания его параметров, множество
	значений которых образует неокторое пространство параметров технического
	состояния.\par
	Таким образом, определение или оценивание технического состояния заключается
	в:
	\begin{enumerate}
		\item указании некоторой точки;
		\item отнесении ее в определенной области пространства параметров
			технического состояния.
	\end{enumerate}

	Параметры технического состояния:
	\begin{itemize}
		\item Измераяемые $X_{\textup{и}}$.\\
			Измеряемыми параметрами технического состояния являются представимые в
			виде значений наблюдаемые или измеряемые показатели или характеристики.
		\item Вычисляемые $X_{\textup{в}}$.\\
			Вычисляемыми параметрами технического состояния являются такие
			характеристики свойсв объекта контроля, которые могут быть вычислены по
			различным алгоритмам с использованием значений измеримых параметров.
	\end{itemize}

	Целью анализа информации, как процесса, являеется получение обобщенных оценок
	совокупности параметров технического состояния, значение которых в явном виде
	указывают:
	\begin{itemize}
		\item степень работоспособности объекта;
		\item место и вид неисправности;
		\item оценки прогнозируемых процессов или явлений с заданным интервалом
			прогноза.
	\end{itemize}

	Задачей анализа имерительной информации является формирование и отнесение
	некоторой точки в области пространства параметров технического состояния, в
	которой сопоставляется допустимое управляющее действие или другая реакция
	пользователя.


\subsection{Формальная постановка задачи распознавания технического состояния}

	$\Omega = {\omega}$ — счетное множество образов подлежащих распознаванию.\par
	$\Omega' = {\omega} \subset \Omega, \quad |\Omega| < +\infty$ \par
	$X = {\chi} = X_{\textup{и}} + X_{\textup{в}}$ — множество параметров
	технического состояния.\par
	$G = {G_i}$ — множество классов образов.\par

	\begin{tikzpicture}
		\node(Osht) at (0,3){$\Omega'$};
		\node(Xu) at (3,3){$X_{\textup{и}}$};
		\node(X) at (3,1.5){$X_{\textup{в}}$};
		\node(O) at (0,0){$\Omega$};
		\node(G) at (3,0){$G$};
		% Omega' to Xu
		\draw[thick,->] (0.5,3) -- (2.5,3) node[pos=0.5, above]{$\psi'$};
		% Omega' to G
		\draw[thick,->] (0.5,2.5) -- (2.5,0.5) node[pos=0.8, above]{$\eta$};
		% Omega to Omega'
		\draw[thick,->] (0,0.5) -- (0,2.5) node[pos=0.5, left]{$\xi$};
		% Omega to Xu
		\draw[thick,->] (0.5,0.5) -- (2.5,2.5) node[pos=0.8, above]{$\psi$};
		% Xu to X
		\draw[thick,->] (3,2.5) -- (3,2.0) node[pos=0.5, right]{$\Psi$};
		% X to G
		\draw[thick,->] (3,1.0) -- (3,0.5) node[pos=0.5, right]{$\varsigma$};
		% Omega to G
		\draw[thick,->] (0.5,0) -- (2.5,0) node[pos=0.5, above]{$\zeta$};
	\end{tikzpicture}

	$\eta : \Omega' \to G$ — отношение обучения.\par
	$\begin{drcases}
		\psi : \Omega \to X_{\textup{и}} \\
		\psi' : \Omega' \to X_{\textup{и}}
	\end{drcases}$ — отношения наблюдения.\par
	$\Psi : X_{\textup{и}} \to X_{\textup{в}}$ — отношение вычислимости.\par
	$\varsigma : X_{\textup{в}} \to G$ — отношение интерпретации.\par
	$\xi : \Omega \to \Omega'$ — отношение обобщения.\par
	$\zeta : \Omega \to G$ — отношение распознавания.\par
	$\zeta : \varsigma * \Psi * \Psi' * \xi \to G$.\par
	$\zeta : \varsigma * \Psi * \xi  \to G$.\par


\chapter{Логические основы систем искусственного интеллекта}
	Появление в математике раздела <<основания математики>> или метаматематики,
	изучающей структуру математических доказательств и математических теорий
	формальными методами обязано открытию ряда парадоксов теории множеств,
	которые заставили усомниться в корректности её построения. Основателем
	метаматематики является	Давид Гильберт (1869~--- 1943).

	В настоящее время достижения метаматематики позволили создать алгоритмы
	докзательства теорем, что легло в основу построения экспертных систем.


\section{Формальные теории}


\subsection{Определение формальных теорий}

	Любая формальная теория оперирует с объектами двух типов:
	\begin{enumerate}
		\item с языком, т.~е. с некоторым множеством высказываний, имеющих смысл;
		\item с совокупностью теорем, являющихся подмножеством языка и состоящих
			из высказываний, истинных для данного множества;
	\end{enumerate}

	Формальная теория определяется кортежем следующего типа:
	$$ FT = \langle U, L, S, T \rangle, $$
	где $ U $~--- алфавит формальной теории,\\
	$ L $~--- язык формальной теории, $ L \subseteq U* $ (множество всех
	подмножеств $ U $,\\
	$ S $~--- подмножество языка, $ S \subset L $ (аксиомы формальной теории),\\
	$ R $~--- правила вывода.\\

	$ U $ определяется как:
	\begin{enumerate}
		\item символы предметных констант ($ \{a, b, c, \dots \} $);
		\item элементы языка (символы предметных переменных, $ \{x, y, z, \dots,
			x_1, y_1, z_1, \dots \} $);
		\item символы функциональных констант ($ \{f_1, f_2, f_3, \dots \} $);
		\item символы предикатных констант ($ \{A, B, C, X, Y, Z, \dots \} $);
		\item рукописные символы ($ \{\mathscr{A, B, C} \} $);
		\item символы логических операторов и кванторов ($ \{\forall, \in, \cup,
			\vee, \oplus, \dots\} $);
		\item вспомогательные символы ($ \{(, ), ;, \dots\} $).\\
	\end{enumerate}

	$ L $ определяется индивидуально с помощью различных конструктивных
	процедур и, как правило, рекурсивно.

	\textit{Отличие формальной теории от формальных систем:} каждой правильно
	построенной формуле может быть сопоставлено некоторое истинностное значение в
	каждой конкретной интерпретации. В формальной теории различают следующие виды
	правильно построенных формул:
	\begin{enumerate}
		\item общезначимые формулы или тавтология~--- тождественно истинные
			относительно всех своих инвариантов;
		\item выполнимые формы~--- в зависимости от интерпретации принимают
			истинное или ложное значение;
		\item противоречивые~--- всегда ложные, во всех интерпретациях.\\
	\end{enumerate}

	$ R $ $ \textit{R}: F_1 \dots F_n \vdash G, $ где $ F_1, \dots, F_n,
	G \in L $\\
	Выводом некоторой формулы $ B $ из формул $ A_1 \dots A_n $ называется
	последовательность формул $ F_1 \dots F_m $, такая, что $ F_m = B $, а любая
	$ F_i (i = 1 \dots m) $ есть:
	\begin{enumerate}
		\item либо аксиома ($ F_i \in S $);
		\item либо одна из исходных формул $ A_1 \dots A_n $;
		\item либо выводится из формул $ F_1 \dots F_{n-1} $ или их подмножества по
			одному из правил вывода;
	\end{enumerate}

	Доказательством формулы $ B $ в формальной теории называют вывод этой формулы
	из пустого множества формул, т.~е. вывод, в котором в качестве формул
	используют только аксиомы.


\subsection{Исчисление высказываний}
	Определение формальной теории исчисления высказываний:
	$$ FT = \langle U, L, S, R \rangle $$
	$ U $:
	\begin{itemize}
		\item[---] переменные высказываний (пропозициональные символы):
			$ A, B, C, \dots $;
		\item[---] символы логических связок: $ \vee, \&, \supset, \neg $;
		\item[---] вспомогательные символы;
	\end{itemize}
	$ L $:
	\begin{itemize}
		\item[---] высказывательная переменная является правильно построенной
			формулой;
		\item[---] если $ A $ и $ B $~--- правильно построенные формулы, то такими
			формулами являются так же и $ A~\vee B, A~\& B, \neg B, \dots $;
	\end{itemize}


\section{Метод резолюций и его использование в системах искусственного
	интеллекта}


\subsection{История поиска процедур доказательства теорем}

	Пролематика: разработка общего алгоритма для проверки общезначимости формулы.
	\begin{itemize}
		\item \textbf{1646–1716 гг.} Лейбниц, первые алгоритмы.
		\item \textbf{XVI век.} Пеано́.
		\item \textbf{1920-е гг.} Д. Гильберт.
		\item \textbf{1930-е гг.} А. Чёрч, А. Тьюринг независимо доказали, что не
			существует общей решающей процедуры (алгиритма), проверяющей
			общезначимость формул в исчислении предикатов. Тем не менее, существуют
			алгоритмы поиска доказательств, которые могут подтвердить общезначимость
			формулы, если она общезначима.
		\item \textbf{1930 г.} Эрбран Жак разработал алгоритм нахожденияs
			интерпретации, которая опровергает заданную формулу, однако, если формула
			общезначима, то такой интерпретации не существует, и алгоритм заканчивает
			работу за конечное число ходов. Этот подход лёг в основу современных
			методов.
		\item \textbf{1960 г.} Гилмор — попытка реализации метода Эрбрана на ЭВМ,
			исходя из тезиса: ``формула общезначима тогда и только тогда, когда её
			отрицание противоречиво``. Программа Гилмора обнаруживала
			противоречивость данной формулы, а значит, подтверждала её
			общезначимость.
		\item \textbf{1960-е гг.} Если отрицание формулы противоречиво, то
			программа обнаруживала этот факт, но была очень громоздкой. Дэвис и
			Патнем доказали чрезмерную громоздкость и неприменимость в практике
			программы Гилмора.
		\item \textbf{1965 г.} Робинсон Джон предложил наиболее эффективный алгоритм поиска
			доказательства, который оказался самым эффективным. Сейчас применяются
			много оптимизированных для разных областей вариаций этого алгоритма:
			\begin{itemize}
				\item семантическая резолюция;
				\item лок-резолюция;
				\item линейная резолюция;
				\item алгоритм британского музея;
				\item и др.
			\end{itemize}
	\end{itemize}



\end{document}
